\chapter{Learning representations theory}
\label{app:ch13}


In the main text, you learned many approaches for representing networks. These methods, while in general rather principled, also have substantial mathematical foundations as to why they are sensible. In particular, when you make assumptions about the network sample that you have, you can in general *prove* that the approaches we described in the main text provide reasonable algorithmic approaches. We outline more in-depth discussions on the sections here:

\begin{enumerate}
    \item Section \ref{app:ch13:mle} discusses maximum likelihood approaches to ER and SBM networks, and the limitations as to why we resort to spectral approaches for RDPGs.
    \item Section \ref{app:ch13:spectral} discusses consistent estimators of latent position matrices and joint matrices.
\end{enumerate}

\section{The basics of maximum likelihood estimation}
\label{app:ch13:mle}
\section{Adjacency Spectral Embedding}
\label{app:ch13:ase}

\subsection{The eigendecomposition allows us to identify latent position matrices from a probability matrix}

The singular value decomposition of the probability matrix $P$ is shown in Figure \ref{sec:ch6:ase:svd}(B).

\begin{figure}
    \centering
    \includegraphics{}
    \caption[\texttt{svd} of a homophilic probability matrix]{\textbf{(A)} the probability matrix, $P$. \textbf{(B)} the eigendecomposition of $P$.}
    \label{fig:ch6:ase:svd}
\end{figure}

If you look very closely at this eigendecomposition, there are some very peculiar facts about it, which we'll get a little more formal about here.

\subsubsection{The eigendecomposition for symmetric positive semi-definite matrices}

Since we are dealing with simple networks, the probability matrix must, by definition, be symmetric. This follows from Section \ref{sec:ch5:ier}, as every $p_{ij} = p_{ji}$ for all pairs of nodes $i$ and $j$. 

Further, remember in Section \ref{sec:ch5:psd_block}, we found that when the block matrix for an $SBM_n(\vec z, B)$ random network was positive semi-definite, the probability matrix would be positive semi-definite, too. In this case, the block matrix is homophilic. Homophilic block matrices are always positive semi-definite, so the probability matrix for a homophilic block matrix of a simple network is both square-symmetric and positive semi-definite.

This means that the probability matrix has an eigendecomposition:

\begin{floatingbox}[h]\caption{The eigendecomposition for real, symmetric matrices}
\label{box:ch6:evd}
The \textit{eigendecomposition} of a real matrix $R$ decomposes it into two real matrices:
\begin{align*}
    R = Q\Lambda Q^\top.
\end{align*}
The matrices are:
\begin{enumerate}
    \item The eigenvectors: A unitary matrix $Q$. Its columns are the eigenvectors $\vec q_i$, and that $Q$ is unitary means that $\vec q_i^\top \vec q_i = 1$ for all $i$, and $\vec q_i \vec q_j = 0$ whenever $i \neq j$. There is one eigenvector for each node $i$, and each eigenvector is a $n$-dimensional column-vector.
    \item The eigenvalues: A diagonal matrix $\Lambda$. Its diagonal entries $\lambda_{i}$ are the eigenvalues, and that $\Lambda$ is diagonal means that all of the off-diagonal entries are $0$. The eigenvalues are non-increasing, in that $\lambda_1 \geq \lambda_2 \geq \hdots \lambda_n$. 
\end{enumerate}
\end{floatingbox}
For such a matrix, using properties of matrix multiplication we can also express the eigendecomposition like this:
\begin{align*}
    R &= \sum_{i = 1}^n \lambda_{i}\vec q_i \vec q_i^\top
\end{align*}
using the intuition built in \cite{Trefethen1997}. 

When $R$ is positive semi-definite, there are two additional qualifiers. The first concerns the sign of the eigenvalues in Remark \ref{box:ch6:evd_nonneg}, in that all of the eigenvalues are going to be non-negative.
\begin{floatingbox}[h]\caption{The eigendecomposition for real, positive semi-definite, symmetric matrices}
\label{box:ch6:evd_nonneg}
When $R$ is real, positive semi-definite, and symmetric, we further obtain that for every $i$, $\lambda_{i}\geq 0$ (the eigenvalues are non-negative).
\end{floatingbox}
So, the eigenvalues are always non-negative.

Another interesting fact about the eigendecomposition that directly relates to this observation is that when the matrix $R$ is positive semi-definite and symmetric, the eigendecomposition can be used to deduce the matrix rank, as explained in Remark \ref{box:ch6:evd_rank}.

\begin{floatingbox}[h]\caption{The eigendecomposition and matrix rank}
\label{box:ch6:evd_rank}
When $R$ is positive semi-definite and symmetric, the number of non-zero eigenvalues corresponds to the matrix rank of $R$.
\end{floatingbox}
So, when $R$ is rank $d$ where $d < n$ ($R$ is not \textit{full-rank}), is symmetric, and positive semi-definite, we have that:
\begin{align*}
    R &= \sum_{i = 1}^n \lambda_{i}\vec q_i \vec q_i^\top \\
    &= \sum_{i = 1}^d \lambda_i \vec q_i \vec q_i^\top + \sum_{i = 1}^{d + 1}\lambda_i \vec q_i \vec q_i^\top \\
    &= \sum_{i = 1}^d \lambda_i \vec q_i \vec q_i^\top. \numberthis \label{eqn:ch6:evd:e1}
\end{align*}
All that we did here was we used properties about the eigenvalues in conjunction with Remarks \ref{box:ch6:evd_nonneg} and \ref{box:ch6:evd_rank}:

\begin{enumerate}
    \item We separated the sum from $1$ to $n$ into the sum from $1$ to $d$ and $d + 1$ to $n$.
    \item In light of Remark \ref{box:ch6:evd_rank}, since $R$ is rank $d$, $d$ of the eigenvalues must be non-zero, and $n - d$ of the eigenvalues must be zero.
    \item Since the eigenvalues are ordered by Remark \ref{box:ch6:evd}, the last $n - d$ of the eigenvalues (corresponding to the eigenvalues from $d + 1$ up to $n$) are the ones that are equal to zero.
\end{enumerate}

So, when $R$ is rank $d$, we can reduce our focus from $Q$ and $\Lambda$ to $Q_d$ and $\Lambda_d$, where:
\begin{align*}
    Q_d &= \begin{bmatrix}
        \uparrow & & \uparrow \\
        \vec q_1 & \hdots & \vec q_d \\
        \downarrow & & \downarrow
    \end{bmatrix},\;\;\;\; \Lambda_d = \begin{bmatrix}
        \lambda_1 & & \\
        & \ddots & \\
        & & \lambda_d
    \end{bmatrix},
\end{align*}
where $Q_d$ is the $n \times d$ matrix consisting of the first $d$ columns of $Q$, and $\Lambda_d$ is the $d \times d$ diagonal matrix consisting of the $d$ non-zero eigenvalues of $R$.

So, combining this with Equation \eqref{eqn:ch6:evd:e1} gives us that:
\begin{align*}
    R &= Q_d \Lambda_d Q_d^\top. \numberthis \label{eqn:ch6:evd:e2}
\end{align*}

Note further that since the eigenvalues are non-negative by Remark \ref{box:ch6:evd_nonneg} and $\Lambda_d$ is diagonal by Remark \ref{box:ch6:evd}, we can easily find a square-root matrix for it:
\begin{align*}
    \Lambda_d = \begin{bmatrix}
        \lambda_{1} & & \\
        & \ddots &  \\
        &&\lambda_{d}
    \end{bmatrix} &= \begin{bmatrix}
        \sqrt{\lambda_{1}} & & \\
        & \ddots &  \\
        && \sqrt{\lambda_{d}}
    \end{bmatrix}\begin{bmatrix}
        \sqrt{\lambda_{1}} & & \\
        & \ddots &  \\
        && \sqrt{\lambda_{d}}
    \end{bmatrix}.
\end{align*}
This matrix, which we will denote using our traditional notation of $\sqrt{\Lambda_d}$, is also diagonal, and its entries are just the square-roots of the non-zero eigenvalues. 

Notice also that since $\Lambda$ is square and diagonal, that $\sqrt{\Lambda_d} = \sqrt{\Lambda_d}^\top$.

Combining this with Equation \ref{eqn:ch6:evd:e2}:
\begin{align*}
    R &= Q_d\Lambda Q^\top \\
    &= Q_d\sqrt{\Lambda_d} \sqrt{\Lambda_d}^\top Q^\top \\
    &= Q_d\sqrt{\Lambda_d}\left(Q_d\sqrt{\Lambda_d}\right)^\top, \numberthis \label{eqn:ch6:evd:e4}
\end{align*}
where we just used the result from Equation \eqref{eqn:ch6:evd:e1}.

\subsection{Conceptualizing the eigendecomposition with $RDPG_n(X)$ random networks}

Since $X$ had $d$ columns, where $d$ was at most $n$, $X$ is a rank $d$ matrix as long as its columns are not scalar multiples of one another. The product of a matrix and its transpose has the same rank as the matrix itself, so $P = XX^\top$ is also rank $d$. Further, $RDPG_n(X)$ random networks have positive semi-definite symmetric probability matrices, so $P$ is rank $d$ where $d$ is the latent dimensionality, positive semi-definite, and symmetric. 

This means that by Equation \ref{eqn:ch6:evd:e4}:
\begin{align*}
    P &= Q_d \sqrt{\Lambda_d}\left(Q_d \sqrt{\Lambda_d}\right)^\top \\
    &= YY^\top,
\end{align*}
where $Y = Q_d \sqrt{\Lambda_d}$. 

It was also the case that $P = XX^\top$. It might feel really easy at this point to conclude that $Y$ and $X$ are equal, because $P = YY^\top$ and $P = XX^\top$. If this were the case, then using the eigendecomposition, we found the latent positions of $P$. Unfortunately, there is a slight caveat.


\subsection{Estimating latent positions with the singular value decomposition}

To recap what we have learned, for an $RDPG_n(X)$ random network, where the latent position matrix $X$ is a $n \times d$-dimensional matrix with $d$ non-redundant columns:
\begin{enumerate}
    \item The probability matrix $P = XX^\top$ is rank $d$,
    \item We can decompose the positive semi-definite probability matrix $P = XX^\top$ into $Q\Lambda Q^\top$, where $Q$ is the matrix whose columns are the eigenvectors of $P$, and $\Lambda$ is the diagonal matrix whose entries are the decreasing, non-negative singular values of $P$,
    \item We can retain {only} the positive singular values and vectors of $P$ and obtain that $P = Q_d \Lambda_d Q_d^\top$ is a decomposition of $P$ into the matrices $Q_d$ which is the first $d$ eigenvectors of $P$ and $\Lambda_d$ which consists of only the first $d$ eigenvalues (and these eigenvalues will be positive), and
    \item The latent position matrix $Y = Q_d \sqrt{\Lambda_d}$ is equivalent to the latent position matrix $X$, up to a rotation.
\end{enumerate}

This is conceptually interesting, but still doesn't really help us when we have a real network $A$, because in practice, we don't know the probability matrix $P$ (we only observe a network $A$ with nodes and edges, not probabilities). 

In practice, $A$ is still symmetric, but it is not necessarily positive semi-definite. For an example, we could consider the adjacency matrix:
\begin{align*}
    A = \begin{bmatrix}
        0 & 1 \\
        1 & 0
    \end{bmatrix}.
\end{align*}
Convince yourself that an eigendecomposition of $A$ is:
\begin{align*}
    Q &= \begin{bmatrix}
        1 \\ 1
    \end{bmatrix},\;\;\;\;\Lambda = \begin{bmatrix}
        1 & \\
        & -1
    \end{bmatrix}.    
\end{align*}
Notice that $\lambda_2 < 0$, so the logic that we built above does not apply. This means that we cannot identify a real square-root matrix for $\Lambda$ (the square-root of $-1$ would be $i$, which is a complex value), so we could not just ``plug in'' the adjacency matrix to our results above to obtain a real latent position matrix. 

For this, we will turn to the \textit{singular value decomposition}, explained in Remark \ref{box:ch6:svd}.

\begin{floatingbox}[h]\caption{The singular value decomposition for square symmetric matrices}
\label{box:ch6:svd}
The \textit{singular value decomposition} of a real matrix $R$ decomposes it into three real matrices:
\begin{align*}
    R = U \Sigma V^\top.
\end{align*}
The matrices are:
\begin{enumerate}
    \item The left singular vectors: A unitary matrix $U$, whose columns $\vec u_i$ are called the left $n$ singular vectors. That it is unitary implies that $UU^\top = I_n$, the identity matrix.
    \item The singular values: A diagonal matrix $\Sigma$, whose entries $\sigma_i$ are called the $n$ singular values. The singular values are arranged such that $\sigma_1 \geq \sigma_2 \geq \sigma_3 \geq ... \geq \sigma_n \geq 0$ (the entries are \textit{decreasing}, and are non-negative).
    \item The right singular vectors: A unitary matrix $V$, whose columns $\vec v_i$ are called the left $n$ singular vectors. That it is unitary implies that $VV^\top = I_n$, the identity matrix.
\end{enumerate}
\end{floatingbox}

Throughout this book, we will notate the singular value decomposition by \texttt{svd}.

For a real symmetric matrix, this means that:
\begin{align*}
    R &= \sum_{i = 1}^n \sigma_i \vec u_i \vec v_i^\top.
\end{align*}
Like the eigendecomposition, for the \texttt{svd} the non-zero singular values will correspond to the rank of the matrix, so for a rank $d$ matrix:
\begin{align*}
    R &= \sum_{i = 1}^d \sigma_i \vec u_i \vec v_i^\top
\end{align*}

So, let's imagine now that $P$ (the probability matrix for an $RDPG_n(X)$ random network) has the decomposition $Q_d \Lambda Q_d^\top$, and an \texttt{svd} is $U \Sigma V^\top$.

\subsubsection{Connecting the eigendecomposition with the \texttt{svd}}

Remember that $Q_d$ is a matrix of eigenvectors, and consequently has orthonormal columns.

This gives us a key observation for our purposes.
\begin{floatingbox}[h]\caption{Orthonormalizing a matrix with orthonormal columns via the Gram-Schmidt process}
\label{box:ch6:gram}
Suppose that $Q_d$ is a matrix with $n$ rows and $d$ columns which are orthonormal. Then there exists a matrix $U$ which has $n$ rows and $n - d$ columns, where the matrix $Q_d^{(U)}$ defined as:
\begin{align*}
    Q_d^{(U)} = \begin{bmatrix}
        Q_d & U
    \end{bmatrix} = \begin{bmatrix}
        \uparrow & & \uparrow &  \uparrow & & \uparrow \\
        \vec q_1 & \hdots & \vec q_d & \vec u_1 & \hdots & \vec u_{n - d} \\
        \downarrow & & \downarrow & \downarrow & & \downarrow
    \end{bmatrix}
\end{align*}
which is square and orthonormal.

This matrix $U$ can be found through the Gram-Schmidt process.
\end{floatingbox}
This key linear algebra result will allow us to make a very important observation for our purposes. Notice that since $Q_d^{(U)}$ is square and orthonormal, that:
\begin{align*}
    Q_d^{(U)}^\top Q_d^{(U)} &= \begin{bmatrix}
        Q_d & U 
    \end{bmatrix}^\top\begin{bmatrix}
        Q_d & U 
    \end{bmatrix} \\
    &= \begin{bmatrix}
        Q_d^\top \\ U^\top
    \end{bmatrix}\begin{bmatrix}
        Q_d & U
    \end{bmatrix} \\
    &= \begin{bmatrix}
        Q_d^\top Q_d & Q_d^\top U \\
        U^\top Q_d & U^\top U
    \end{bmatrix} \numberthis \label{eqn:ch6:svd:e1} \\
    &= I_n = \begin{bmatrix}
        I_d & 0_{d \times n - d} \\
        0 _{n - d \times d} & I_{n - d}
    \end{bmatrix}, \numberthis \label{eqn:ch6:svd:e2}
\end{align*}
so simply by matching up the corresponding blocks of Equation \eqref{eqn:ch6:svd:e1} and Equation \ref{eqn:ch6:svd:e2}:
\begin{enumerate}
    \item $Q_d Q_d^\top = I_d$,
    \item $Q_d^\top U = 0_{d \times n - d}$,
    \item $U^\top Q_d = 0_{n - d \times d}$,
    \item $UU^\top = I_{n - d}$.
\end{enumerate}


Using these facts we can easily deduce an interesting corollary:
\begin{corollary}[The matrix $U$ whose columns orthonormalize the columns of $Q_d$ is not unique]
If $Q_d$ is a matrix with $d$ orthonormal columns and $U$ is a matrix whose columns orthonormalize the columns of $Q_d$, $U$ is not unique. This means that there is another matrix $V$ whose columns also orthonormalize the columns of $Q_d$. Consequently, for such a matrix $V$, $Q_d^{(V)}$ is also square and orthonormal.
\end{corollary}
We'll do a very short proof of this result, so that you can convince yourself this is the case. This is the most math-intensive result in this book, so bear with us. 
\begin{proof}Suppose that $W$ is a $n - d$ dimensional rotation matrix, and let $V = UW$. Then:
\begin{enumerate}
    \item $Q_d Q_d^\top = I_d$,
    \item $Q_d^\top V = 0_{d \times n - d}$, because:
    \begin{align*}
        Q_d^\top V &= Q_d U W \\
        &= (Q_d U)W \\
        &= 0_{d \times n - d} W, 
    \end{align*}
    which is because the columns of $U$ orthonormalize the columns of $Q_d$. Then, since the product of a matrix and the zero-matrix is the zero-matrix:
    \begin{align*}
       Q_d^\top V &= 0_{d \times n - d}
    \end{align*}
    \item $V^\top Q_d = 0_{n - d \times d}$, because:
    \begin{align*}
        V^\top Q_d &= (Q_d^\top)^\top \\
        &= 0_{n - d \times d}^\top \\
        &= 0_{d \times n - d}
    \end{align*}
    Which follows directly from above, and
    \item $VV^\top = I_{n - d}$, because:
    \begin{align*}
        V^\top V &= \left(UW\right)^\top UW \\
        &= W^\top U^\top U W \\
        &= W^\top I_{n - d}W \\
        &= W^\top W,
    \end{align*}
    because a matrix times the identity returns the matrix itself, and therefore:
    \begin{align*}
        V^\top V &= I_{n - d}
    \end{align*}
    because the matrix $W$ is a $n - d$-dimensional rotation matrix, so $W^\top W = I_{n - d}$.
\end{enumerate}
Therefore, $Q_d^{(V)}$ where $V$ is any rotation of $U$ is also square and orthonormal, because $Q_d^{(V)}^\top Q_d^{(V)} = I_n$.
\end{proof}

This next result is where the light bulbs will start to go off:
\begin{corollary}[Equivalence of top $d$ left and right singular vectors]
Suppose that $P$ is a square, symmetric, and positive semi-definite matrix which is rank $d$. Then the top $d$ left and right singular vectors will be equal, and the remaining singular vectors will be equal up to a rotation.
\end{corollary}
\begin{proof}
Remember that if $P$ is rank $d$ and is square, symmetric, and positive semi-definite, that it has the eigendecomposition $P = Q\Lambda Q^\top$ and the matrix decomposition $Q_d \Lambda_d Q_d^\top$. Remember that by definition, $\Lambda_d$ is a diagonal matrix whose entries $\lambda_i$ are positive. 

Extend $Q_d$ to form the square orthonormal matrices $Q_d^{(U)}$ and $Q_d^{(V)}$, where $U$ and $V$ are equal up to a rotation, using Remark \ref{box:ch6:gram}.

Let $\Sigma$ be a diagonal matrix whose first $d$ entries are $\sigma_i = \lambda_i$ for the first $d$ non-zero eigenvalues of $\Lambda$, and $0$ for the remaining $n - d$ diagonal entries.

Then using Equation \eqref{eqn:ch6:evd:e1}, $P$ can be expressed as:
\begin{align*}
    P &= \sum_{i = 1}^d \lambda_i \vec q_i \vec q_i^\top \\
    &= \sum_{ i =1}^d \lambda_i \vec q_i \vec q_i^\top + 0 \\
    &= \sum_{i = 1}^d \lambda_i \vec q_i \vec q_i^\top + \sum_{i = d + 1}^n 0 \\
    &= \sum_{i = 1}^d \sigma_i \vec q_i \vec q_i^\top + \sum_{i = d + 1}^n \sigma_i \vec u_{i - d}\vec v_{i - d}^\top,
\end{align*}
which is because $\sigma_i = \lambda_i > 0$ for all $i \leq d$, and $\sigma_i = 0$ for all $i > d$. 

Note that this is equivalent to writing that:
\begin{align*}
    P &= Q_d^{(U)}\Sigma Q_d^{(V)}^\top,
\end{align*}
where $Q_d^{(U)}$ and $Q_d^{(V)}$ are square and orthonormal, and $\Sigma$ is a diagonal matrix whose entries are non-negative.

Then $P = Q_d^{(U)}\Sigma Q_d^{(V)}$ is a \texttt{svd} of $P$. 
\end{proof}
\section{Theoretical considerations for spectral embeddings}
\label{app:ch13:spectral}


\bibliographystyle{vancouver}
\bibliography{references}