\chapter{Applications when you have networks}
\label{sec:ch8}

In this section, we'll proceed much the same as the preceding section. We'll learn how to learn from networks, but particularly in the case where you have two networks. When you have two networks, there are some nuances to analytical procedures that arise, and many special questions that you can ask between the two networks. We'll cover a few here.

\begin{enumerate}
    \item Section \ref{sec:ch8:twosample} provides methods for testing whether a pair of RDPGs are different, the two-sample testing problem.
    \item Section \ref{sec:ch8:twosamplesbm} shows a special-case of the two-sample testing problem when the networks are SBMs.
    \item Section \ref{sec:ch8:gm} introduces the graph matching problem for aligning two networks.
    \item Section \ref{sec:ch8:vnviasgm} shows how we can identify similar nodes across a pair of networks using strategies that we developed for graph matching.
\end{enumerate}

\section{Two-sample testing for networks}
\paragraph*{Co-authored with Sambit Panda}
\label{sec:ch8:twosample}

In Section \ref{sec:ch6:multinet}, we learned several techniques for dealing with multiple networks. When you find networks in your data, a fundamental question that you might have is how to determine whether the networks differ. This problem is another instance of a two-sample test, much like the two-sample test that we learned about in Section \ref{sec:ch7:testing:twosample}. Remember that a \textit{two-sample test} was a test to determine whether two samples of data were different. 

To do this, we'll return to the alien and human brain networks example, from Case Study \ref{box:ch6:multinet:ex}. There are $n=100$ brain areas, which represent the nodes of the network. This time, we find a third group of aliens, and their brain structure is much more similar to humans than our original group of aliens (they are still homophilic). However, this group of aliens tends to have heterogeneities in the node degrees: nodes with lower indices have a higher degree than nodes with higher indices, whereas the human brains tend to have more homogeneous structurings.

Let's imagine that we have one human brain, and one alien brain:

\begin{lstlisting}[style=python]
from graspologic.simulations import sbm
import numpy as np
from graphbook_code import dcsbm, generate_dcsbm_pmtx, \
                           generate_sbm_pmtx

n = 100  # the number of nodes
M = 8  # the total number of networks
# human brains have homophilic block structure
Bhum = np.array([[0.2, 0.02], [0.02, 0.2]])
# alien brains add degree-correction
theta_alien = np.tile(np.linspace(1.7, 0.3, 50), 2)

# generate 4 human and alien brain networks
A_human, z = sbm([n // 2, n // 2], Bhum, return_labels=True)
A_alien = dcsbm(z, theta_alien, Bhum)

Phum = generate_sbm_pmtx(z, Bhum)
Palien = generate_dcsbm_pmtx(z, theta_alien, Bhum)
\end{lstlisting}

Plots of the networks are shown in Figure \ref{fig:ch8:twosample:ex}, where their underlying probability matrices are shown in \textbf{(A)} and \textbf{(C)}. Note that the degree-correction factor for the alien brain network has led to the nodes with lower indices in each community to have higher probabilities of being connected to other nodes. Samples of the human and alien brain networks are \textbf{(B)} and the alien brain network in \textbf{(D)}. It is a little bit less obvious that the networks are different when we look at the network samples. 

Just like when we were flipping coins we could not immediately conclude that two coins were different just because they produced a different number of heads in a given number of flips, it would be unwise to conclude that the networks differ fundamentally (they have different probability matrices) just because the samples look different. 

\begin{figure}[h]
    \centering
    \includegraphics[width=0.7\linewidth]{applications/ch8/Images/ts_ex.png}
    \caption[Human and alien brains]{\textbf{(A)} probability matrix underlying human brain network, \textbf{(B)} brain network from human, \textbf{(C)} probability matrix underlying alien brain network, \textbf{(D)} brain network from alien.}
    \label{fig:ch8:twosample:ex}
\end{figure}

For this reason, it is important for us to build out our two-sample testing procedures, so that we can determine whether the networks differ.

\subsection{Two-sample tests and random networks}

In this section, we'll go a slightly different direction from the last time we saw two-sample testing. When we have two random networks $\mathbf A^{(1)}$ and $\mathbf A^{(2)}$, our first thing that we need to do is determine appropriate models for our networks. If we make the assumption that these networks are $IER_n\left(P^{(m)}\right)$ random networks from Section \ref{sec:ch5:ier}, which are the broadest class of independent-edge random network models, remember that the probability matrix encodes the differences between the two random networks. Therefore, our question is:
\begin{align*}
    H_0 : P^{(1)} = P^{(2)} \text{ against }H_A : P^{(1)} \neq P^{(2)}, \numberthis \label{eqn:ch8:twosample:2shypo_prob}
\end{align*}
or that the edge probability matrices are the same (under the null hypothesis) against that the edge probability matrices differ (under the alternative hypothesis). This is exactly the same question that we would ask about determining whether two coins are different, but generalized to $n \times n$ probability matrices.

Remember, however, that the $IER_n\left(P^{(m)}\right)$ random networks are difficult to deal with (directly) when you only have a single network sample. This means that testing the hypotheses in Equation \eqref{eqn:ch8:twosample:2shypo_prob} is fairly impractical when you only have two networks, because you only have one sample $a_{ij}^{(m)}$ to base your estimate of $p_{ij}^{(m)}$ on. 

\subsection{Latent positition testing}
\label{sec:ch8:twosample:lpt}
For this reason, it is fairly typical to make further assumptions about the network structure. As you learned in Section \ref{sec:ch6:ase}, a typical assumption that we can make is that the networks are $RDPG_n\left(X^{(m)}\right)$ random networks, with latent position matrices $X^{(m)}$. Remember that your latent position matrix $X^{(m)}$ fully describes the probability matrix $P^{(m)}$, so it would be (almost) equivalent to just compare the latent positions directly:
\begin{align*}
    H_0 : X^{(1)} = X^{(2)} \text{ against }H_A : X^{(1)} \neq X^{(2)}, \numberthis \label{eqn:ch8:twosample:2shypo_lpm:dumb}
\end{align*}
or would it? Remember that for a network with a latent position matrix $X^{(m)}$, that for any $d \times d$ rotation matrix $W$, by the non-identifiability problem in Section \ref{sec:ch6:spectral:nonidentifiable}:
\begin{align*}
    P^{(m)} &= X^{(m)}X^{(m)}^\top = X^{(m)}WW^\top X^{(m)}^\top
\end{align*}
This means that even if $P^{(1)}$ and $P^{(2)}$ were identical, we could define their latent position matrices differently, and still end up with the same random network, by having $X^{(2)} = X^{(1)}W$ for some rotation matrix $W$. For this reason, we'll make our hypothesis ``robust'' to this challenge, and test:
\begin{align*}
    H_0 :& X^{(1)} = WX^{(2)}\text{ for some rotation matrix $W$} \\
    H_A :& X^{(1)} \neq WX^{(2)} \text{ for any rotation matrix $W$}. \numberthis \label{eqn:ch8:twosample:2shypo_lpm:smart}
\end{align*}
In words, our null hypothesis is that the latent positions are identical up to a rotation (for some possible rotation matrix), and our alternative hypothesis is that the latent positions are not rotations of one another (for any possible rotation matrix). 

\subsubsection*{Learning rotations from the data}

From Section \ref{sec:ch6:ase}, we know that we can obtain estimates of $X^{(m)}$ by using the \texttt{ase}, where $\hat X^{(m)} = \texttt{ase}\left(A^{(m)}\right)$. How do we account for the potential that $X^{(1)}$ and $X^{(2)}$ are rotations of one another?

\paragraph*{The orthogonal Procrustes problem}

To estimate a possible rotation, we will need to set up an optimization problem. Our goal is to find the best possible rotation of $X^{(2)}$ onto $X^{(1)}$. 

\begin{floatingbox}[h]\caption{Concept: The Frobenius norm}
\label{box:ch7:twosample:frobnorm}
The Frobenius norm can be thought of as a generalization of the Euclidean norm from Concept \ref{def:ch6:se:eucl_dist} to matrices. If $A$ is a $n \times m$ matrix with entries $a_{ij}$, the Frobenius norm is:
\begin{align*}
    \|A\|_F &= \sqrt{\sum_{i = 1}^n \sum_{j = 1}^m a_{ij}}
\end{align*}
We can use the Frobenius norm to define a distance. If $A$ and $B$ are two matrices with the same number of rows ($n$) and columns ($m$), their Frobenius distance is:
\begin{align*}
    \|A - B\|_F &= \sqrt{\sum_{i = 1}^n \sum_{j = 1}^m (a_{ij} - b_{ij})^2}
\end{align*}
\end{floatingbox}

Using Concept \ref{box:ch7:twosample:frobnorm}, we can define what we mean by the ``best'' possible rotation: the rotation where the Frobenius distance between $X^{(1)}$ and $X^{(2)}$ are minimized. We can describe our goal as:
\begin{align*}
    \text{find }W \text{ where } \left\|X^{(1)} - X^{(2)}W\right\|_F\text{ is minimized}.
\end{align*}
We can equivalently write this goal as:
\begin{align*}
    \hat W &= \text{argmin}_W \left\|X^{(1)} - X^{(2)}W\right\|_F.\numberthis \label{eqn:ch8:twosample:opp}
\end{align*}
The idea is that if $X^{(1)}$ and $X^{(2)}$ are rotations of one another, we can (at least, in theory) for the $d \times d$ rotation matrix $\hat W$ that properly aligns them. If such a rotation exists, the function that we are minimizing $\left\|X^{(1)} - X^{(2)}\hat W\right\|_F$ will have a value of $0$ when we find the correct rotation (because the matrices will be identical). The problem of solving for optimal rotations in Equation \eqref{eqn:ch8:twosample:opp} is known as the \textit{orthogonal Procrustes problem}.

In practice, we only have estimates $\hat X^{(1)}$ and $\hat X^{(2)}$, so we will not find the perfect rotation of $\hat X^{(1)}$ onto $\hat X^{(2)}$ (even if such a rotation exists for $X^{(1)}$ and $X^{(2)}$). Therefore, when we minimize $\left\|\hat X^{(1)} - \hat X^{(2)}W\right\|_F$, we will just want the $\hat W$ that makes them the most similar (in terms of the Frobenius distance). If $X^{(1)}$ and $X^{(2)}$ are the same, we would expect that $\left\|\hat X^{(1)} - \hat X^{(2)}\hat W\right\|_F$ would therefore take a relatively small value.

Programatically, we can code this up as:

\begin{lstlisting}[style=python]
from scipy.linalg import orthogonal_procrustes
from graspologic.models import RDPGEstimator
import numpy as np

d = 2
# estimate latent positions for alien and human networks
Xhat_human = RDPGEstimator(n_components=d).fit(A_human).latent_
Xhat_alien = RDPGEstimator(n_components=d).fit(A_alien).latent_
# estimate best possible rotation of Xhat_alien to Xhat_human by 
# solving orthogonal procrustes problem
W = orthogonal_procrustes(Xhat_alien, Xhat_human)[0]
observed_norm = np.linalg.norm(Xhat_human - Xhat_alien @ W, ord="fro")
\end{lstlisting}

This gives us the value shown in Figure \ref{fig:ch8:twosample:lpt}(A), in gray. A relatively small value, relative what exactly?

\begin{figure}
    \centering
    \includegraphics[width=\linewidth]{applications/ch8/Images/ts_ldt_ex.png}
    \caption[Latent position and latent distribution testing]{\textbf{(A)} The observed statistic from the data (gray) compared to a histogram from the $r=100$ parametric replicates (black). \textbf{(B)} A comparison of two buckets of coins, where each coin has a different probability of landing on heads. \textbf{(C)} The observed statistic from the data (gray) compared to a histogram from the $r=1000$ replicates (black).}
    \label{fig:ch8:twosample:lpt}
\end{figure}

\subsubsection{Generating a null distribution through parametric bootstrapping}

If you recall from Section \ref{sec:ch7:testing:twosample}, we learn with hypothesis tests by looking at how well the data reflect the null hypothesis; that is, that the estimated latent positions (computed from the data) are identical (up to a rotation). 

When you have estimates computed from the data, ideally, you will be able to determine directly (by the nature of the data and the assumptions that you make) exactly how ``anomalous'' the difference that you observe is relative what you would expect under the null hypothesis. In the case of the contingency tables from Section \ref{sec:ch7:testing:twosample}, the Fisher's exact test used the assumptions that we made (that the two samples were independent binary events) to compute how ``anomalous'' the contingency table that we observed was. 

In this case, however, without more assumptions we can't quite compute this value exactly. In the interest of keeping assumptions to a minimum, a popular strategy is to use what is known as a \textit{bootstrap}. A \textit{bootstrap} is a technique which attempts to quantify the sampling distribution of some quantity (a \textit{statistic}) that is computed from the data. The \textit{sampling distribution} refers to the distribution of the statistic $\left\|\hat X^{(1)} - \hat X^{(2)}\hat W\right\|_F$, where $\hat X^{(1)}$, $\hat X^{(2)}$, and $\hat W$ are computed from the data.

The way that we will do this is that we will use the parameter $\hat X^{(1)}$ as the latent position matrix for two new random networks, $\mathbf A'$ and $\mathbf A''$. These two random networks have an identical latent position matrix: $\hat X^{(1)}$. Then, we will generate two new samples of these random networks with the same latent position matrix:

\begin{lstlisting}[style=python]
from graspologic.simulations import rdpg

def generate_synthetic_networks(X):
    """
    A function which generates two synthetic networks with
    same latent position matrix X.
    """
    A1 = rdpg(X, directed=False, loops=False)
    A2 = rdpg(X, directed=False, loops=False)
    return A1, A2

Ap, App = generate_synthetic_networks(Xhat_human)
\end{lstlisting}

We then use these two new networks, and again estimate latent position matrices for these:

\begin{lstlisting}[style=python]
def compute_latent(A, d):
    """
    A function which returns the latent position estimate
    for an adjacency matrix A.
    """
    return RDPGEstimator(n_components=d).fit(A).latent_

Xhat_p = compute_latent(Ap, d)
Xhat_pp = compute_latent(App, d)
\end{lstlisting}

Finally, just like we aligned $\hat X^{(1)}$ and $\hat X^{(2)}$ via the ``optimal rotation'' $\hat W$, we will align $\hat X'$ and $\hat X''$ via the ``optimal rotation'' $\hat W'$. Then, we recompute our statistic of interest $\left\|\hat X' - \hat X''W'\right\|_F$ using these new samples, where the underlying latent positions were truly identical:

\begin{lstlisting}[style=python]
def compute_norm_orth_proc(A, B):
    """
    A function which finds the best rotation of B onto A,
    and then computes and returns the norm.
    """
    R = orthogonal_procrustes(A, B)[0]
    return np.linalg.norm(A - B @ R)

norm_null = compute_norm_orth_proc(Xhat_p, Xhat_pp)
\end{lstlisting}


We keep repeating this process again and again, and over time, we gradually get some idea of what $\left\|\hat X^{(1)} - \hat X^{(2)}\hat W\right\|_F$ would look like if the true latent positions $X^{(1)}$ and $X^{(2)}$ were identical up to a rotation $W$. This is known as a \textit{parametric bootstrap}. It is called \textit{parametric} because we are using the assumption that the networks are RDPGs to generate what $\hat X^{(1)}$ and $\hat X^{(2)}$ would look like under the null hypothesis (that they are identical up to a rotation). 

The statistics that we compute from the parametric null replicates (the statistics under the null hypothesis) are indicated by the histogram on Figure \ref{fig:ch8:twosample:lpt}(A), in black. Notice that the statistic is less extreme than the one that we observed from the data, shown in gray. 

The below code will repeat this procedure $r=100$ times (the number of \textit{replicates} of the bootstrap resampling), and keep track of the computed statistics under the null hypothesis:

\begin{lstlisting}[style=python]
def parametric_resample(A1, A2, d, nreps=100):
    """
    A function to generate samples of the null distribution under H0
    using parametric resampling.
    """
    null_norms = np.zeros(nreps)
    Xhat1 = compute_latent(A1, d)
    for i in range(0, nreps):
        Ap, App = generate_synthetic_networks(Xhat1)
        Xhat_p = compute_latent(Ap, d)
        Xhat_pp = compute_latent(App, d)
        null_norms[i] = compute_norm_orth_proc(Xhat_p, Xhat_pp)
    return null_norms

nreps = 100
null_norms = parametric_resample(A_alien, A_human, 2, nreps=nreps)
\end{lstlisting}

What we see is that $||\hat X^{(p)} - \hat X^{(r)}\hat W||_{F}$ is much larger than the almost all of the values of $|| \hat X' - \hat X''\hat W'||_{F}$ that we calculated. We will use this to estimate a $p$-value, and we will say that the $p$-value of $H_0$ against $H_A$ is the fraction of times that when the underlying latent positions were equal (for each replicate of our parametric bootstrap), the value of the statistic that we calculated from the observed data exceeded the statistic that we calculated from the replicate. We add one to the numerator and denominator, since we observed one instance of a value at least as big as that in the observed data: the observed data itself. 

This means our $p$-value is:

\begin{lstlisting}[style=python]
pval = ((null_norms >= observed_norm).sum() + 1)/(nreps + 1)
print("estimate of p-value: {:.3f}".format(pval))
# estimate of p-value: 0.010
\end{lstlisting}

We then repeat this process, but we use $A^{(2)}$ as our reference point instead of $A^{(1)}$. The overall $p$-value is the maximum of the two $p$-values produced with this procedure.

This is called the \textit{latent position test}, and it is implemented directly by \texttt{graspologic}. Note that the $p$-value that you obtain from this process might differ every time you run the test, since there is randomness in your generation process of the $A'$s and the $A''$s for every time you repeated the comparison. Making the number of repetitions larger by setting \texttt{n\_bootstraps} to a higher value will tend to yield more stable $p$-value estimates for when we estimate $p$-values using resampling techniques:
\begin{lstlisting}[style=python]
from graspologic.inference import latent_position_test

nreps = 50 # the number of null replicates
lpt = latent_position_test(A_human, A_alien, n_bootstraps = nreps, n_components=d, workers=-1)
print("estimate of p-value: {:.5f}".format(lpt[1]))
# estimate of p-value: 0.00100
\end{lstlisting}
The $p$-value is low, and below a typical decision threshold of $\alpha = 0.05$. This means we have evidence to reject the null hypothesis in favor of the alternative: the latent position matrix for a human brain network differs from the latent position matrix for an alien brain network (by more than just a rotation).

Next, let's see what happens if we have two networks with the same underlying latent position matrices. We'll make a second network from a human, and use the latent position test to determine whether the networks have identical latent position matrices (up to a rotation):
\begin{lstlisting}[style=python]
# generate a new human brain network with same block matrix
A_human2 = sbm([n // 2, n // 2], Bhum)

lpt_hum2hum = latent_position_test(A_human, A_human2, n_bootstraps=nreps, n_components=d, workers=-1)
print("estimate of p-value: {:.5f}".format(lpt_hum2hum[1]))
# estimate of p-value: 0.65347
\end{lstlisting}
The $p$-value is relatively large, so with $\alpha = 0.05$, we fail to be able to reject the null hypothesis. We conclude that our data does not support that our two human networks have different underlying latent position matrices (which is great, because they are the same). The latent position test is described in \cite{Tang2017Apr}.

\subsection{Latent distribution testing}

To motivate a latent distribution test, we're going to return to yet another coin example. Imagine that you and a friend do not each have a single coin like the examples that you have seen so far, but you each have a container of $300$ coins. Each of these $300$ coins can be identified as the coin $\mathbf x_i^{(y)}$ or $\mathbf x_i^{(f)}$, your $i^{th}$ coin or your friend's $i^{th}$ coin. 

The outcomes $\mathbf x_i^{(y)}$ and $\mathbf x_i^{(f)}$ are random. These coins all have different probabilities of landing on heads, $\mathbf p_i^{(y)}$ and $\mathbf p_i^{(f)}$, again for your coin and your friend's coin, respectively, and the probabilities that the *coins* land on heads is random too! Let's say, for sake of example, that your coins have probabilities of landing on heads that tend to be right around $\frac{3}{8}$, but your friend's coins have probabilities of landing on heads right around $\frac{5}{8}$:

\begin{lstlisting}[style=python]
ncoins = 300  # the number of coins in each container

# the probabilities of your coins landing on heads
piy = np.random.beta(a=3, b=5, size=ncoins)

# the probabilities of your friend's coins of landing on heads
pif = np.random.beta(a=5, b=3, size=ncoins)
\end{lstlisting}
You each grab a single coin at random from your containers, and flip them and observe whether they land on heads or tails, the outcomes $x_i^{(y)}$ and $y_i^{(f)}$, respectively. Since the probabilities each coin lands on heads is random here, we can't actually compare the probabilities. But what we can compare are characteristics about your coins' and your friend's coins' probabilities. 

You could ask, for instance, whether the average probability that your coins land on heads is less than the average probability that your friend's coins land on heads. The difference is that we are asking about parameters that underly the random probabilities of the random coins in the experiment, and not about the random probabilities themselves.

Much the same, we could assume that not only are the latent positions for humans and aliens different, but these might be random too, just like the probabilities of the coins in the example you just learned about. What we mean when we say that the latent position matrices are random, we mean that the latent positions for each node, the vectors $\vec x_i^{(1)}$ and $\vec x_i^{(2)}$ for each of the $n$ total nodes, are not necessarily fixed quantities. There might be random variables that underly these too: $\mathbf{\vec x}_i^{(1)}$ and $\mathbf{\vec x}_i^{(2)}$. 

We won't need to go too in-depth here, but the basic idea is that the latent position vectors for each node, $\mathbf{\vec x}_i^{(1)}$ and $\mathbf{\vec x}_i^{(2)}$, have underlying parameters as well, just like the random probabilities for your coins and your friend's coins tended to be around $\frac{3}{8}$ and $\frac{5}{8}$ respectively. All of the characteristics that determine how the human or alien latent position vectors can be realized are governed by functions called the distributions of the latent position vectors. We use the symbol $F^{(1)}$ and $F^{(2)}$, respectively, to denote the distribution of the humans' latent position vectors and the aliens' latent position vectors, respectively. When we ask questions about whether $\mathbf P^{(1)}$ and $\mathbf P^{(2)}$ differ, now our question no longer boils down to just checking whether the latent positions themselves are different, but whether the distributions of the latent positions are different.

did before, we will make null and alternative hypotheses for these situations. Your null hypothesis is going to be that $H_0 : F^{(1)} = F^{(2)}W$, which means that the distributions of the latent positions are the same (like before, we allow for a possible $W$). The alternative hypothesis is going to be that the latent positions for the Moors and the Moops have a different distribution for any possible rotation, $H_A : F^{(1)} \neq F^{(2)}W$.

The general idea is that we will assume as little as possible about the distributions for the latent positions. Two good ways to do this are using an approach called distance correlation \cite{Szekely2007Dec} or multiscale generalized correlation (MGC) \cite{Vogelstein2019Jan}. You can do these very easily using \texttt{graspologic}, through the \texttt{latent\_distribution\_test} function:

\begin{lstlisting}[style=python]
from graspologic.inference import latent_distribution_test

nreps = 50
approach = 'dcorr'  # the strategy for the latent distribution test
ldt_dcorr = latent_distribution_test(A_hum, A_alien, test=approach, metric="euclidean", n_bootstraps=nreps, workers=-1)
\end{lstlisting}

The relevant things to look at for the latent distribution is again the $p$-value:

\begin{lstlisting}[style=python]
print("p-value of H0 against HA: {:.4f}".format(ldt_dcorr[1]))
\end{lstlisting}

The null replicates against the observed test statistic are shown in Figure \ref{fig:ch8:twosample:lpt}(C). For more details on this approach, check out \cite{Alyakin2020}.

\subsection{When would you use the latent distribution test versus the latent position test?}
\label{sec:ch8:twosample:lpt_vs_ldt}

The latent distribution test is called non-parametric because it does not make the assumption that the networks are RDPGs with fixed latent position matrices like we did in the latent position test. In this sense, the latent distribution test tends to be a little bit more general than the latent position test, and will tend to be more \textit{conservative}. In statistics, a \textit{conservative} test is a testing procedure which tends to err on the side of caution: the $p$-values, and consequently your conclusions, will generally tend to err towards shying away from rejecting the null hypothesis. When we obtain results in favor of $H_A$ with conservative tests (such as a small $p$-value), we tend to have a little bit more confidence that our results hold up to scrutiny, which is a good thing if you are trying to convince people you are correct and that your assumptions did not dictate your conclusions.

Moreover, the latent position test assumes that the two networks have node matching: node $1$ in the first network has the same interpretation as node $1$ in the second network, and so on for all $n$ nodes in the network. The latent distribution test does not make this rather limiting assumption: you can perform a latent distribution test when the nodes differ in both interpretation (nodes don't need to be matched) or in number (you could compare networks without the same number of nodes entirely). This makes the latent distribution test a bit more flexible, in general, than the latent position test.

\paragraph*{Generalizing to other network models}

The results that we learned in this section generalize directly to any random network models that can be described as $RDPG_n(X)$ random networks: this includes any network model where the probability matrix has a positive semi-definite structure. As you learned in Section \ref{sec:ch5:psd_block}, this includes many other network models, such as DCSBMs and RDPGs with positive semi-definite block matrices. 

The utility of these techniques, however, likely generalizes beyond just the $RDPG_n(X)$ random networks, and probably includes gRDPG random networks, too. This is because the primary piece of information that is needed to motivate the techniques described here was that $\hat X^{(1)}$ and $\hat X^{(2)}$ were reasonable (unbiased and consistent) estimates of the latent position matrices $X^{(1)}$ and $X^{(2)}$ (up to a rotation). As you learned in Section \ref{sec:ch6:ase:whyuse}, spectral embedding techniques can produce similarly reasonable estimates of latent position matrices for gRDPG random networks too. While there is no direct theory that asserts that the latent position and latent distribution tests are appropriate for gRDPGs, it is pretty reasonable to expect them to apply in for gRDPGs (with potentially non positive semi-definite probability matrices) too. For the latent position test, a key modification would be that the parametric bootstrap would need to consider samples from gRDPG random networks, and not $RDPG_n(X)$ random networks (with the rest of the logic applying as-is).

\subsection{Read on for more information}

Appendix \ref{app:ch12:rdpg} discusses the foundational assumptions for the latent distribution test, the \textit{a posteriori} Random Dot Product Graph. Appendix \ref{app:ch13:spectral} discusses the implications of latent distribution testing with respect to spectral embeddings.

\newpage
\section{Two-sample testing for SBMs}
\label{sec:ch8:twosamplesbm}

In Section \ref{sec:ch8:twosample}, we saw a situation where we had two networks which we thought could be effectively characterized with RDPG. Further, we knew that the communities were the same across both networks: that is, we knew that community $1$ in the first network was the same as community $1$ in the second network, so on and so forth all the way up to community $K$. In this situation, we found that we could test whether the latent positions for the underlying RDPGs are the same; that is, whether $H_0: X^{(1)} = X^{(2)}W$ against $H_A: H^{(1)} \neq X^{(2)}W$. We called this the two-sample hypothesis test for RDPGs. 

What if we can take this a step further, however, and we can say that the networks are realizations of SBMs? How can we check whether the block matrices are the same? If you remember from Section \ref{sec:ch5:psd_block:lpm_fromsbm}, we learned that the latent position matrix for positive semi-definite block matrices is just a function of the community assignment vector and the block matrix. Therefore, if two networks have the same community assignment vectors but their positive semi-definite block matrices are unequal, their latent position matrices are unequal. Therefore, we could use the machinery that we developed in Section \ref{sec:ch8:twosample} to test whether the block matrices are unequal.

However, we can approach this question more directly for SBMs: we can test whether the block matrices are unequal directly. In Case Study \ref{box:ch8:twosampsbm:ex}, we develop an example we will work with for this section.

\begin{floatingbox}[h]\caption{Case Study: Traffic patterns}
We have two networks which summarize the traffic patterns between $n=100$ towns (represented by the nodes in our network) across $K=3$ states (represented by the communities in our network). The first 45 towns are in New York (NY, the first community), the second 30 towns are in New Jersey (NJ, the second community), and the third 25 towns are in Pennsylvania (PA, the third community). Our goal is to examine traffic pattern differences during the daytime compared to night time.

For a month, we measure the number of drivers who commute from one town to the other in a specified time window, and if more than $1,000$ drivers regularly make this commute, we add an edge between the pair of towns. In general, we know that people tend to commute more frequently within their state, so the probabilities that an edge exists between a pair of towns in the same state exceeds the probabilities that an edge exists between a pair of towns which are not in the same state. Now, here's the twist: we have measured the first network between 7 AM and 7 PM (covering the bulk of the work day), and the second network between 7 PM and 7 AM (covering the bulk of night time). We know that a lot of people in New Jersey tend to commute to new York for the work day, so we the probability of an edge existing between a New Jersey town and a New York town are higher during the day than the night.

We don't think that driving patterns themselves really change too much otherwise, but we do think that the probability of an edge existing is about $50\%$ higher during the daytime for all pairs of communities in the network.
\end{floatingbox}

Let's generate the community assignment vector and the block matrices for the day and night time. The day time network will be network $(1)$, and the night time network will be network $(2)$:
\begin{lstlisting}[style=python]
import numpy as np
from graspologic.simulations import sbm
ns = [45, 30, 25]  # number of students

# z is a column vector indicating which state each
# town is in
z = np.array(["NY" for i in range(0, ns[0])] + 
              ["NJ" for i in range(0, ns[1])] +
              ["PA" for i in range(0, ns[2])] )

Bnight = np.array([[.3, .15, .15], [.15, .3, .15], [.15, .15, .3]])
Bday = Bnight*1.5  # day time block matarix is 50% more than night

# people tend to commute from New Jersey to New York during the day
Bday[0, 1] = .4; Bday[1,0] = .4

Anight = sbm(ns, Bnight)
Aday = sbm(ns, Bday)
\end{lstlisting}

The block matrices, and the two network samples, are shown in Figure \ref{fig:ch8:twosampsbm:ex}.

\begin{figure}
    \centering
    \includegraphics[width=\linewidth]{applications/ch8/Images/twosamp_sbm_ex.png}
    \caption[Two-sample SBM comparison]{\textbf{(A)} the block matrix for night time, \textbf{(B)} the block matrix for day time, \textbf{(C)} the adjacency matrix for night time, \textbf{(D)} the adjacency matrix for day time.}
    \label{fig:ch8:twosampsbm:ex}
\end{figure}

\subsection{Testing whether the block matrices in an SBM are different}

Based on what we learned above, we know ahead of time that the block matrices for the SBMs are different. However, how can we actually test this? Well, let's start by being clear about what we mean by "different". To make this a little big more mathemattical, we'll introduce some new variables for the block matrices during the day time ($B^{(1)}$) and at night time ($B^{(2)}$) clearly. The block matrices are:

\begin{align*}
    B^{(1)} &= \begin{bmatrix}
    b^{(1)}_{11} & b^{(1)}_{12} & b^{(1)}_{13} \\
    b^{(1)}_{21} & b^{(1)}_{22} & b^{(1)}_{23} \\
    b^{(1)}_{31} & b^{(1)}_{32} & b^{(1)}_{33}
    \end{bmatrix}; \;\;\; B^{(2)} = \begin{bmatrix}
    b^{(2)}_{11} & b^{(2)}_{12} & b^{(2)}_{13} \\
    b^{(2)}_{21} & b^{(2)}_{22} & b^{(2)}_{23} \\
    b^{(2)}_{31} & b^{(2)}_{32} & b^{(2)}_{33}
    \end{bmatrix}
\end{align*}
The hypothesis we want to test is the null hypothesis that the block matrices are the same, $H_0: B^{(1)} = B^{(2)}$, against the alternative hypothesis that the block matrices are different, $H_A: B^{(1)} \neq B^{(2)}$. For a matrix, remember that two matrices are equal if all of the entries are identical, and two matrices are unequal if at least one of the entries are unequal. We can reformulate the null and alternative hypotheses with this logic.

For the null hypothesis, $H_0: B^{(1)} = B^{(2)}$, the statement is therefore equivalent to saying that for all pairs of communities $k$ and $l$, $b^{(1)}_{kl} = b^{(2)}_{kl}$. We will write each of these statements down as individual hypotheses for all pairs of communities, using the convention $H_{0, kl}: b_{kl}^{(1)} = b^{(2)}_{kl}$. 

The null hypothesis $H_0$ is therefore equivalent to saying that for every pair of communities $k$ and $l$, $H_{0,kl}$ is true. 

For the alternative hypothesis, $H_A: B^{(1)} \neq B^{(2)}$, the statement is therefore equivalent to saying that for at least one pair of communities $k$ and $l$, $b^{(1)}_{kl} \neq b^{(2)}_{kl}$. We will write down each of these statements as well as individual hypotheses for all pairs of communities, using the convention $H_{A, kl} : b_{kl}^{(1)} \neq b^{(2)}_{kl}$. The alternative hypothesis $H_A$ is therefore equivalent to saying that for at least one pair of communities $k$ and $l$, that $H_{A,kl}$ is true.

Now that we have broken a statement about two matrices down into numerous statements about two probabilities, we have almost completed our job. As it turns out, we have already seen the way we will test this, back in testing for differences in Section \ref{sec:ch7:testing}. Seeking to test whether a pair of block probabilities between communities $k$ and $l$ are the same, $H_{0,kl}$, against whether the pair of block probabilities between communities $k$ and $l$ are different, $H_{A, kl}$, is the two-sample testing problem, like Section \ref{sec:ch7:testing:twosample}. We addressed this problem before using Fisher's exact test, and that is what we will do here, too.

Remember that with Fisher's exact test, for two probabilities that we want to compare, we construct a contingency table. In this case, we will have $\binom K 2 + K$ contingency tables (one for each unique pair of communities, plus the on-diagonal edges where both nodes have the same community), which are shown in Table \ref{tab:ch8:twosampl_sbm:cont}.

\begin{table}[h]
    \centering
    \begin{tabular}{c|c| c}
         & $(1)$, night time & $(2)$, day time  \\
         \hline
         Number of edges & $a$ & $b$ \\
         Number of non-edges &$c$ & $d$
    \end{tabular}
    \caption[Two-sample SBM contingency table.]{An example of a contingency table for testing for a difference in the block matrices between two SBMs. $\binom K 2$ of these contingency tables are constructed, corresponding to the contingency tables between nodes in community $k$ and $l$ for all possible pairs of communities.}
    \label{tab:ch8:twosampl_sbm:cont}
\end{table}

In the table, entry $a$ is the total number of edges between nodes of community $k$ with nodes of community $l$ in the daytime network, and $b$ is the total number of edges between nodes of community $k$ with nodes of community $l$ in the daytime network. The entry $c$ the total number of potential edges between nodes of community $k$ with nodes of community $l$ in the day time network that do not exist (the number of adjacencies with an adjacency of zero), and the entry $d$ is the total number of potential edges between nodes of community $k$ with nodes of community $l$ in the night time network that do not exist. 

We implement this using \texttt{numpy} and \texttt{scipy}, just like we did before, but this time for each pair of communities. To identify which adjacency matrix entries correspond to a given pair of communities, we use \texttt{np.outer}. Since the network is simple, we also exclude the on-diagonal entries, and avoid double counting by only looking at the upper-right triangle of the adjacency matrix:

\begin{lstlisting}[style=python]
from scipy.stats import fisher_exact

K = 3
Pvals = np.empty((K, K))
# fill matrix with NaNs
Pvals[:] = np.nan

# get the indices of the upper triangle of Aday
upper_tri_idx = np.triu_indices(Aday.shape[0], k=1)
# create a boolean array that is nxn
upper_tri_mask = np.zeros(Aday.shape, dtype=bool)
# set indices which correspond to the upper triangle to True
upper_tri_mask[upper_tri_idx] = True

for k in range(0, K):
    for l in range(k, K):
        comm_mask = np.outer(z == (k+1), z == (l + 1))
        table = [[Aday[comm_mask & upper_tri_mask].sum(), Anight[comm_mask & upper_tri_mask].sum()],
                 [(Aday[comm_mask & upper_tri_mask] == 0).sum(), (Anight[comm_mask & upper_tri_mask] == 0).sum()]]
        Pvals[k,l] = fisher_exact(table)[1]
\end{lstlisting}

This gives us a matrix of $p$-values, whose $p$-values correspond to tests of $H_{0, kl}$ against $H_{A, kl}$ for all pairs of communities $k$ and $l$. However, this still does not give us an answer to our original question, whether the block matrices $B^{(1)}$ and $B^{(2)}$ were the same against the alternative that the block matrices $B^{(1)}$ and $B^{(2)}$ were different.

\subsubsection*{Adjusting for multiple comparisons}

When performing multiple statistical tests, we run into the multiple hypothesis correction problem. Let’s imagine that we have $5000$ coins, and each of these coins has a true probability of landing on heads of $0.5$. We flip each coin 
$500$ times, and for each coin $i$, we estimate the probability that the coin lands on heads by just counting the number of heads and dividing by $500$. For each coin, we want to test whether the probability that the coin lands on heads is different from $0.5$. In symbols, we want to test $H_0^{(i)} : p^{(i)} = 0.5$ against $H_A^{(i)}: p^{(i)} \neq 0.5$. For this problem, an appropriate statistical test is known as the Binomial test, which is described in Appendix \ref{app:ch14:hypotest_intro}. For the purposes of this section, all that you need to know is that it is a way of testing whether the data supports (or does not support) that a probability is equal to a fixed constant. Let's run our experiments:

\begin{lstlisting}[style=python]
import numpy as np
from graspologic.simulations import er_np
import seaborn as sns
from scipy.stats import binom_test

ncoins = 5000 # the number of networks
p = 0.5  # the true probability
n = 500  # the number of flips

# the number of heads from each experiment
experiments = np.random.binomial(n, p, size=ncoins)

# perform binomial test to see if the number of heads we obtain supports that the
# true probabiily is 0.5
pvals = [binom_test(nheads_i, n, p=p) for nheads_i in experiments]
\end{lstlisting}


\begin{figure}
    \centering
    \includegraphics[width=\linewidth]{applications/ch8/Images/twosampsbm_mc.png}
    \caption{\textbf{(A)} a histogram of the $p$-values before multiple comparisons adjustment, \textbf{(B)} a histogram of the $p$-values after adjusting for multiple comparisons, by preserving the FWER via Holm-Bonferroni correction.}
    \label{fig:ch8:twosampsbm:mc}
\end{figure}
A histogram of the $p$-values from all $5000$ tests (one for each coin) are shown in Figure \ref{fig:ch8:twosampsbm:mc}(A). In general, we will correctly reject the alternative hypothesis, and accept the null hypothesis. This is great, since $p_i$ is, in fact, $0.5$ as specified by the null hypothesis $H_{0}^{(i)}$. However, we notice something particularly strange: For a portion of the tests, we are, in-fact, wrong. We obtain many $p$-values which are under our decision threshold of $\alpha = 0.05$, and would incorrectly report that the alternative hypothesis is correct, and $p_i$ is not $0.5$. This is wrong, and extremely problematic! Remember that the $p$-value was defined as the probability that the null hypothesis (here, that $p_i = 0.5$) would be incorrectly rejected in favor of the alternative hypothesis (here, that $p_i \neq 0.5$). In fact, if the null hypothesis were true, we would expect to see $p$-values of at most $\alpha$ being reported about $\alpha$ of the time (this assumes all of the coins are independent, but even when they are not all independent, we still obtain a similarly shocking conclusion). This means that with $n=5000$ tests and $\alpha = 0.05$, we would expect to be wrong about $n\cdot \alpha = 50$ times.

Basically, the problem here is that while each test individually only has a $\alpha = 0.05$ chance of incorrectly rejecting the null hypothesis (when it is true), by running multiple tests, we have increased the familywise error rate to be well north of $\alpha = 0.05$. The \textit{familywise error rate} (FWER) is the probability that we made an error and incorrectly rejected the null hypothesis for any one of the hypothesis tests we ran. You can read more about this problem here \cite{Ryan1959Jan}.

As practicians, it feels like it would be pretty problematic to report an analysis and know that an arbitrary fraction of your conclusions are wrong just by random chance. For this reason, a focus of statistics in recent decades has been the development of methods which, in effect, inflate the $p$-values based on the number of tests that we perform, so that we run into this issue at much lower rate than $\alpha$ of the time. These strategies are collectively known as \textit{multiple comparisons adjustments}. We won't go into too many details with how multiple comparisons adjustments are performed, but in general given a list of $p$-values \texttt{pvals}, you can adjust the $p$-values using the \texttt{multipletests()} method from \texttt{statsmodels}. This will give you protection for this ``multiple comparisons'' issue in your analyses, and make you a more conservative statistician. Like in Section \ref{sec:ch8:twosample:lpt_vs_ldt}, conservative statistical approaches are approaches which tend to err on the side of caution.

Let's see what happens when we adjust our $p$-values here using a popular method called Bonferroni-Holm adjustment \cite{Holm1979}:
\begin{lstlisting}[style=python]
from statsmodels.stats.multitest import multipletests

alpha = 0.05  # the desired alpha of the test
_, adj_pvals, _, _ = multipletests(pvals, alpha=alpha, method="holm")
\end{lstlisting}

A histogram of the $p$-values after adjustment is shown in Figure \ref{fig:ch8:twosampsbm:mc}(B). After adjusting for multiple comparisons, we end up with all of the $p$-values being $1$. Therefore, we never incorrectly reject the null hypothesis anymore.

In general, for multiple hypothesis correction, we recommend using Holm-Bonferroni correction, which is encoded with the parameter \texttt{method="holm"}. We recommend the use of the Holm-Bonferroni approach because it ensures that our $p$-value produced by using multiple statistical tests controls the FWER with no requirements as to the problem we glossed over previously, the dependence of the hypotheses being tested. This may give us adjusted $p$-values that are a little higher than several other methods (such as the popular Benjamini-Hochberg procedure), but in general we prefer to err on the side of caution when reporting scientific discoveries and do not want to spend effort investigating hypothetical dependencies amongst our hypothesis tests.

Let's adjust the $p$-values that we estimated for the pairwise community comparisons for our networks, upper-right symmetrize the resulting matrix of $p$-values (because we only ran tests on the upper-right triangle of the adjacency matrix), and then plot the result:
\begin{lstlisting}[style=python]
from graspologic.utils import symmetrize

Pvals_adj = multipletests(Pvals.flatten(), method="holm")[1].reshape(K, K)
Pvals_adj = symmetrize(Pvals_adj, method="triu")
\end{lstlisting}

\begin{figure}
    \centering
    \includegraphics[width=\linewidth]{applications/ch8/Images/twosamp_sbm_pvals.png}
    \caption[$p$-values for differences between block matrices]{\textbf{(A)} $p$-values for test of a difference between the block matrices, \textbf{(B)} $p$-values for test of a difference between the block matrices, after rescaling.}
    \label{fig:ch8:twosampsbm:pval_mtx}
\end{figure}

The matrix of $p$-values (after adjustment for multiple comparisons) is shown in Figure \ref{fig:ch8:twosampsbm:pval_mtx}(A). The $p$-values are all extremely small, so we can conclude that the block matrices $B^{(1)}$ and $B^{(2)}$ differ for all entries. The $p$-value for the test of $H_0$ that the block matrices are identical against $H_A$ that the block matrices are different can be taken to be the minimum of all of the adjusted $p$-values:

\begin{lstlisting}[style=python]
pval_dif = Pvals_adj.min()
print("p-value of block matrix difference: {:.4f}".format(pval_dif))
# p-value of block matrix difference: 0.0000
\end{lstlisting}
which is so small that it rounds to $0$. With $\alpha = 0.05$ as usual, we reject the null hypothesis that the block matrices are identical in favor of the alternative that the block matrices are different. 

\subsection{Testing whether the block matrices in an SBM are multiples of one another}

Back in Section \ref{sec:ch5:prop:rndeg} we learned about a useful summary statistic for random networks, known as the expected density. The expected network density could be written as:
\begin{align*}
\mathbb E[density(\mathbf A)] &= \frac{\sum_{i = 1}^n \mathbb E[\mathbf d_i]}{n(n - 1)} 
\end{align*}
Which could be thought of as the expected average degree of each node in the network, divided by the maximum possible degree each node could have. 

As it turns out, the network density plays an extremely large role in virtually every property of networks which we estimate in machine learning, including the block matrices. Remember from Concept \ref{box:ch5:dcsbm:exp_deg}, that for a $SBM_n(\vec z, B)$ random network, the average degree for a node in community $k$ was:
\begin{align*}
    \mathbb E[\mathbf d_i ; z_i = k] &= \sum_{l \neq k} n_l b_{lk} + (n_k - 1)b_{kk}
\end{align*}
so it is clear that the expected node degree is a function of the block matrix. As the expected density is a function of the expected degrees, the expected density is a function of the block matrix, too. We could also reverse this argument, and establish a relationship between the block matrices and the expected density in the network.

This means that if the expected densities are different, the block matrices will, by default, also be different.

For instance, in our example, we know that the probability of an edge existing between two cities is, in general, about $50\%$ higher during the daytime compared to the night time. We don't want to have our answer just be a product of the fact that there were just more edges in the daytime network. Rather, we want to find the topological difference between the two networks; that is, that the day time driving patterns between New Jersey and New York towns went above and beyond the $50\%$ increase we otherwise saw. For this reason, we revamp our hypothesis a little bit.

If you remember, our hypothesis that we ran above was the null hypothesis $H_0: B^{(1)} = B^{(2)}$ that the two block matrices are the same against the alternative $H_A: B^{(1)} \neq B^{(2)}$ that the block matrices differ. We will change this up a little bit. Now, our null hypothesis becomes $H_0: B^{(1)} = \alpha\cdot B^{(2)}$ that the two block matrices are the same up to a rescaling against $H_A: B^{(2)} \neq \alpha\cdot B^{(2)}$, that they differ even after possible rescalings. How do we interpret this?

In this case, the value $\alpha$ is chosen to be the difference in the expected network densities between the day time and night time networks, the quantity $\alpha = \frac{p^{(1)}}{p^{(2)}}$. In practice, what we use is an estimate of this quantity, $\hat \alpha = \frac{\hat p^{(1)}}{\hat p^{(2)}}$, where $\hat p^{(1)}$ is the network density of the night time network and $\hat p^{(2)}$ is the network density of the day time network. Accepting the alternative hypothesis here means that the block matrices of the day and night time network are not simply multiples of each other.

\begin{floatingbox}[h]\caption{Case Study: Bilateral symmetry in fruit fly brains}
\label{box:ch8:twosamp_sbm:conn}
It is fairly well-established that the brain of human beings is asymmetric: different sides of your brain are responsible for related, but distinct, functions \cite{Springer2001Sep}. This was definitively established in the 1860s by a scientist named Paul Broca, who discovered that patients with brain lesions to the same area of the left frontal portion of the brain experienced similar symptoms of speech loss. That two patients with similar lesions experienced the same symptoms provided evidence that brain function was \textit{localized}, in that different areas of the brain were responsible for different functions. Related observations made by psychologists led to conclusions that took this a step further, that different hemispheres of the brain were responsible for different functions.

In a recent paper \cite{Pedigo2022Nov}, investigators were able to connect a similar ``localization of connectivity'' in the wiring of the brain itself. The investigators used connectomes of a fruit fly for the left and right hemispheres, where the nodes were individual neurons and the edges were individual connections (called \textit{synapses}) between these neurons. Using the methods that we explained above, they were able to show that some types of cells in the brain had totally different densities of edges between the two hemispheres. After accounting for these hemisphere density disparities, several different cell types had similar connectivity patterns.
\end{floatingbox}

We address this problem very similarly to above, instead using the chi-squared test for non-unity probability ratios from \cite{Dunnett1977Dec, Chan2003Jan, Miettinen1985Apr}. We covered the chi-squared test in Section \ref{sec:ch7:modelselect} when we were covering model selection. We implement this using the bilateral connectome package \cite{neurodata2023Feb}, again showing the $p$-value matrix to visualize which combinations of communities are substantially different. This approach is described in \cite{Pedigo2022Nov}:

\begin{lstlisting}[style=python]
from pkg.stats import stochastic_block_test, combine_pvalues

stat, _, misc = stochastic_block_test(Anight, Aday,
    labels1=z, labels2=z, density_adjustment=True)
Pval_adj_rescaled = np.array(misc["corrected_pvalues"])
\end{lstlisting}

This new matrix of $p$-values is shown in Figure \ref{fig:ch8:twosampsbm:pval_mtx}(B), This shows that after we adjust the block matrix for changes in network density, the difference in the block probability for traveling between New York and New Jersey is still significant. This means that the density-adjusted block matrices are different, since the minimum corrected $p$-value is less than $0.05$.


The interpretation here is that, after adjusting for density, we can still reject the null hypothesis in favor of the alternative that the block matrices for day and night time are different, and that this difference can be accounted for by different traffic patterns between New York and New Jersey in day time versus night time. For an example of the use of these strategies in practice, check out Case Study \ref{box:ch8:twosamp_sbm:conn}.


\newpage


\section{The graph matching problem}
\label{sec:ch8:gm}
\paragraph*{Co-authored with Ali Saad-Eldin}
At this point, we've spent a good deal of time deciding how to determine whether or not two networks are distinct in some way. We first saw this example (in the general case) when the networks were samples of $RDPG_n(X)$ random networks, wherein we could use the latent position and latent distribution test to see whether the underlying random networks were different. Next, we saw how in the special case that the networks were samples of $SBM_n(\vec z, B)$ random networks, we could use Fisher's exact tests and modified versions of chi-squared tests to investigate whether the block matrices differed.

% In this section, we'll turn this around with the \textit{graph matching problem}. The \textit{graph matching problem} is the problem of uncovering similarities between a pair of networks. Let's take a look at a running example in Case Study \ref{box:ch8:gm:ex}.

In this section, we'll turn this around with the \textit{graph matching problem}. The \textit{graph matching problem} is the problem of finding a good match between the indices of the nodes of two networks.

\begin{floatingbox}[h]\caption{Case Study: Graph matching across social networks}
\label{box:ch8:gm:ex}
You work at Facebook and Twitter, but there’s been a terrible incident on the Twitter end. All Twitter users’ names and handles have been somehow been deleted! Your bosses have tasked you with recovering the lost information. How might you go about doing this? Luckily, you have a great resource at your disposal: the Facebook social network. You know all Facebook users and who they are friends with, and since you’ve only lost the Twitter usernames, you can still figure out which unnamed Twitter users follow each other. You decide to use the Facebook network connectivity data to re-label the Twitter social network. Alternatively, you can say the we are ``aligning’’ or ``matching'' Twitter to Facebook. \\

In the two social networks above, each user is a node and an edge exists if two users are friends. We'll define the Facebook and Twitter networks as $F$ and $T$ respectively, with associated adjacency matrices $A^{(F)}$ and $A^{(T)}$. Aligning the nodes of two networks is known as graph matching, because we are matching the node indices of one network (or, graph) to another. This can also be thought of as a mapping; that is, based on the neighbors of a node in $F$, you find a node in $T$ with the most similar neighborhood structure, then give the two nodes the same index. In other words, one of our Twitter users will be assigned the user name of the Facebook user with the most connections in common. This is done for the whole network, with the end result being that overall the structure is best preserved.
\end{floatingbox}

\subsection{Brute Forcing is Infeasible}
If you were to pick randomly, there would be an infeasibly large number of ways to match the nodes of two networks. In fact, for network pairs with $n$ nodes, there are $n!$ (factorial) possible mappings. For example, when $n=100$, there are more than $10^{157}$ possible matchings. So, our task is to figure out which mapping is best without the computationally gargantuan task of checking each one. For a thorough review of the history of different ways of approaching this problem, see \cite{Livi2013Aug}.

\subsection{Defining a similarity metric}

First, we need a metric. We want this metric to be small when two networks, $f\left(A^{(1)}, A^{(2)}\right)$, are similar, and large when they are not. For graph matching, we use the Frobenius distance from Concept \ref{box:ch7:twosample:frobnorm}:
\begin{align*}
    f\left(A^{(1)}, A^{(2)}\right) &= \left\|A^{(1)} - A^{(2)}\right\|_F.
\end{align*}

To understand this functionally, consider the best possible case where where the two networks are identical: $A=B$. 

\begin{align*}
A^{(1)} = 
\begin{array}{cc} &
\begin{array}{ccc} 0 & 1 & 2 \end{array}
\\
\begin{array}{cc}
0 \\
1 \\
2 \end{array}
&
\left[
\begin{array}{ccc}
0 & 1 & 1\\
1 & 0 & 1\\
1 & 1 & 0\end{array}
\right]\end{array}
\quad \quad
A^{(2)} = 
\begin{array}{cc} &
\begin{array}{ccc} 0 & 1 & 2 \end{array}
\\
\begin{array}{ccc}
0 \\
1 \\
2 \end{array}
&
\left[
\begin{array}{ccc}
0 & 1 & 1\\
1 & 0 & 1\\
1 & 1 & 0\end{array}
\right]\end{array}
\\
A^{(1)}-A^{(2)} =
\begin{array}{cc} &
\begin{array}{ccc} 0 & 1 & 2 \end{array}
\\
\begin{array}{ccc}
0 \\
1 \\
2 \end{array}
&
\left[
\begin{array}{ccc}
0 & 0 & 0\\
0 & 0 & 0\\
0 & 0 & 0\end{array}
\right]\end{array}
\\
\left\|A^{(1)} - A^{(2)}\right\|_F^2 = 0.
\end{align*}
As we see above, the difference will be a matrix of all zeros, and taking the squared Frobenius norm will then yield $f\left(A^{(1)}, A^{(2)}\right) = 0$. This is because all of the element-wise differences $a_{ij}^{(1)} - a_{ij}^{(2)}$ are just zero, and hence both their square (and sum) will also be zero. Below we remove one edge from $A^{(2)}$:

\begin{align*}
A^{(1)} = 
\begin{array}{cc} &
\begin{array}{ccc} 0 & 1 & 2 \end{array}
\\
\begin{array}{ccc}
0 \\
1 \\
2 \end{array}
&
\left[
\begin{array}{ccc}
0 & 1 & 1\\
1 & 0 & 1\\
1 & 1 & 0\end{array}
\right]\end{array}
\quad \quad
A^{(2)} = 
\begin{array}{cc} &
\begin{array}{ccc} 0 & 1 & 2 \end{array}
\\
\begin{array}{ccc}
0 \\
1 \\
2 \end{array}
&
\left[
\begin{array}{ccc}
0 & 1 & 1\\
1 & 0 & 0\\
1 & 0 & 0\end{array}
\right]\end{array}
\\
A^{(1)} - A^{(2)} =
\begin{array}{cc} &
\begin{array}{ccc} 0 & 1 & 2 \end{array}
\\
\begin{array}{ccc}
0 \\
1 \\
2 \end{array}
&
\left[
\begin{array}{ccc}
0 & 0 & 0\\
0 & 0 & 1\\
0 & 1 & 0\end{array}
\right]\end{array}
\\
\left\|A^{(1)} - A^{(2)}\right\|_F^2 = 2.
\end{align*}

Because these networks are unweighted and undirected, we are effectively counting the total number of disagreements in the adjacency matrices between $A^{(1)}$ and $A^{(2)}$.

\subsubsection*{Graph Matching Small Networks}


Instead of (1) and (2), let's use $T$ and $F$ for Twitter and Facebook. our two networks, $A^{(T)}$ and $A^{(F)}$, have four nodes each: $\{1, 2, 3, 4\}$ for $A^{(T)}$, and $\{a, b, c, d\}$ for $A^{(F)}$. In this case, the nodes represent people within the social networks, and the edges represent whether two people are connected on the social networking site. In this case, we will assume we have a node correspondence:
\begin{enumerate}
    \item Person $0$ on Twitter is person $a$ on Facebook,
    \item Person $1$ on Twitter is person $b$ on Facebook,
    \item Person $2$ on Twitter is person $c$ on Facebook,
    \item Person $3$ on Twitter is person $d$ on Facebook.
\end{enumerate}

\paragraph*{Node order is irrelevant if node correspondence is respected}

The corresponding adjacency matrices of the two networks are equal to each other when the nodes are laid out for $A^{(T)}$ as $\{0, 1, 2, 3\}$, and when the nodes are laid out for $A^{(F)}$ as $\{a,b,c,d\}$. The two networks are illustrated, with their nodes laid out with respect to the node correspondence, in Figure \ref{fig:ch8:gm:ex}(A).

\begin{align*}
A^{(F)} = 
\begin{array}{cc} &
\begin{array}{cccc} 0 & 1 & 2 & 3 \end{array}
\\
\begin{array}{cccc}
0 \\
1 \\
2 \\
3 \end{array}
&
\left(
\begin{array}{cccc}
0 & 1 & 1 & 0\\
1 & 0 & 0 & 1\\
1 & 0 & 0 & 1\\
0 & 1 & 1 & 0\end{array}
\right)\end{array}
\quad \quad
A^{(T)} = 
\begin{array}{cc} &
\begin{array}{cccc} a & b & c & d \end{array}
\\
\begin{array}{ccc}
a \\
b \\
c \\
d \end{array}
&
\left(
\begin{array}{cccc}
0 & 1 & 1 & 0\\
1 & 0 & 0 & 1\\
1 & 0 & 0 & 1\\
0 & 1 & 1 & 0\end{array}
\right)\end{array} \\
\left|A^{(F)} - A^{(T)}\right| = \left[
\begin{array}{cccc}
0 & 0 & 0 & 0\\
0 & 0 & 0 & 0\\
0 & 0 & 0 & 0\\
0 & 0 & 0 & 0\end{array}
\right]
\end{align*}

\begin{figure}[h]
    \centering
    \includegraphics[width=\linewidth]{applications/ch8/Images/gm_ex.png}
    \caption[Social network $4$ node graph matching example]{\textbf{(A)} the social networks, with the nodes aligned. \textbf{(B)} the social networks, with the nodes mis-aligned. \textbf{(C)} the social networks, with the nodes mis-aligned, but with the topology of the network identical.}
    \label{fig:ch8:gm:ex}
\end{figure}
    
We can see above that $f(A_T, A_B) = 0$. By ordering the nodes, we're claiming that there is some kind of match between the Twitter and Facebook nodes. Ordering the adjacency matrices under this claim will give us a low edge disagreement (here, just zero). 

If we permute the node indices of both networks in the same way, we can still find a low edge disagreement. Let's reorder the nodes for Twitter and Facebook as $\{2, 0, 1, 3\}$ and $\{c, a, b, d\}$ respectively. We will call these new adjacency matrices $A_T'$ and $A_F'$ respectively:

\begin{align*}
A^{(T)}' = 
\begin{array}{cc} &
\begin{array}{cccc} 2 & 0 & 1 & 3 \end{array}
\\
\begin{array}{cccc}
2 \\
0 \\
1 \\
3 \end{array}
&
\left[
\begin{array}{cccc}
0 & 1 & 0 & 1 \\
1 & 0 & 1 & 0\\
0 & 1 & 0 & 1\\
1 & 0 & 1 & 0
\end{array}
\right]\end{array}
\quad \quad
A^{(F)}' = 
\begin{array}{cc} &
\begin{array}{cccc} c & a & b & d \end{array}
\\
\begin{array}{ccc}
c \\
a \\
b \\
d \end{array}
&
\left[
\begin{array}{cccc}
0 & 1 & 0 & 1 \\
1 & 0 & 1 & 0\\
0 & 1 & 0 & 1\\
1 & 0 & 1 & 0
\end{array}
\right]\end{array} \\
\left|A^{(T)}' - A^{(F)}'\right| = \left[
\begin{array}{cccc}
0 & 0 & 0 & 0\\
0 & 0 & 0 & 0\\
0 & 0 & 0 & 0\\
0 & 0 & 0 & 0\end{array}
\right]
\end{align*}

Even though the ordering of the nodes is different, like goes with like: the first node for Twitter is node $2$ and the first node for Facebook is node $c$, the second node for Twitter is $0$ and the second node for Facebook is node $a$, so on and so forth. The node orderings preserve the correspondence between the nodes of Twitter with the nodes of Facebook.

\paragraph*{Networks rarely come pre-ordered}

The ordering of a network's nodes in an adjacency is arbitrary, which can often make it hard to tell whether two networks are the same. 

Let's say we had an arbitrary ordering of the nodes for Facebook, which was a little bit different from the one which we just saw: The nodes are ordered $\{a, b, d, c\}$ instead of $\{a, b, c, d\}$. Twitter's nodes are still ordered as $\{0, 1, 2, 3\}$. This is illustrated in Figure \ref{fig:ch8:gm:ex}(B). We use $A^{(F)``}$ to denote the new ordering. The adjacency matrices are no longer equal:

\begin{align*}
A^{(T)} = 
\begin{array}{cc} &
\begin{array}{cccc} 0 & 1 & 2 & 3 \end{array}
\\
\begin{array}{cccc}
0 \\
1 \\
2 \\
3 \end{array}
&
\left[
\begin{array}{cccc}
0 & 1 & 1 & 0\\
1 & 0 & 0 & 1\\
1 & 0 & 0 & 1\\
0 & 1 & 1 & 0\end{array}
\right]\end{array}
\quad \quad
A^{(F)}'' = 
\begin{array}{cc} &
\begin{array}{cccc} a & b & d & c \end{array}
\\
\begin{array}{cccc}
a \\
b \\
d \\
c \end{array}
&
\left(
\begin{array}{cccc}
0 & 1 & 0 & 1\\
1 & 0 & 1 & 0\\
0 & 1 & 0 & 1\\
1 & 0 & 1 & 0\end{array}
\right)\end{array} \numberthis \label{eqn:ch8:gm:fb_perm:e1}\\
\left|A^{(T)} - A^{(F)}''\right| = \left[
\begin{array}{cccc}
0 & 0 & 1 & 1\\
0 & 0 & 1 & 1\\
1 & 1 & 0 & 0\\
1 & 1 & 0 & 0\end{array}
\right]
\end{align*}
    
Our similarity metric changes as well: $f\left(A^{(T)}, A^{(F)}''\right) = 8$, since there are $8$ entries which are different in the adjacency matrices between $A^{(T)}$ and $A^{(F)}''$. This might seem a bit high, but remember - the network is undirected, so adjacency disagreements are effectively counted twice (a single edge disagreement for an edge $(i,j)$ also yields a disagreement for edge $(j, i)$, since the adjacency matrix is symmetric). By comparing the nodes of Twitter and Facebook with the nodes misaligned, we have effectively broken the node correspondence between the nodes of Twitter and Facebook. 

Let's explore how to manipulate our adjacency matrices such that we can find alignments that match well.

\begin{floatingbox}[h]\caption{Low numbers of edge disagreements do not imply node correspondence}

Let's imagine that we mix the ordering of the nodes for Facebook a third time, instead using $\{b, a, d, c\}$, to yield another adjacency matrix $A^{(F)}'''$, illustrated in Figure \ref{fig:ch8:gm:ex}(C). Note that the adjacency matrices are identical for $A^{(F)}'''$ and $A^{(T)}$, despite the face that the nodes for Facebook are not ordered with respect to the node correspondence of the Twitter network:

\begin{align*}
A^{(T)} = 
\begin{array}{cc} &
\begin{array}{cccc} 0 & 1 & 2 & 3 \end{array}
\\
\begin{array}{cccc}
0 \\
1 \\
2 \\
3 \end{array}
&
\left[
\begin{array}{cccc}
0 & 1 & 1 & 0\\
1 & 0 & 0 & 1\\
1 & 0 & 0 & 1\\
0 & 1 & 1 & 0\end{array}
\right]\end{array}
\quad \quad
A^{(F)}''' = 
\begin{array}{cc} &
\begin{array}{cccc} b & a & d & c \end{array}
\\
\begin{array}{ccc}
b \\
a \\
d \\
c \end{array}
&
\left[
\begin{array}{cccc}
0 & 1 & 1 & 0\\
1 & 0 & 0 & 1\\
1 & 0 & 0 & 1\\
0 & 1 & 1 & 0\end{array}
\right]\end{array} \\
\left|A^{(T)} - A^{(F)}'''\right| = \left[
\begin{array}{cccc}
0 & 0 & 0 & 0\\
0 & 0 & 0 & 0\\
0 & 0 & 0 & 0\\
0 & 0 & 0 & 0\end{array}
\right]
\end{align*}

This illustrates that the adjacency matrices can be identical even when we do not have a correspondence of the nodes between the two networks, which presents a challenge and limitation for the graph matching problem that you should be aware of. Formally, this means that if two networks are identical (up to an ordering of the nodes), there may be multiple ways to orient the nodes of the networks in which you obtain no edge disagreements. Stated another way, we could have multiple different networks where the topology (as indicated by the adjacency matrix) is the same.
\end{floatingbox}
\subsection{Permutation Matrices}

Permutation matrices are commonly used as a method to move around the rows and columns of a square matrix. A \textit{permutation matrix} is a matrix where, for every row and column, exactly one entry has a value of one. 
\subsubsection*{$P^\top A$ moves the rows}

Let's consider a matrix $A$ where all entries of the first row have a value of one, all entries of the second row have a value of two, all entries of the third row have a value of three, and all entries of the fourth row have a value of four. This matrix is shown in Figure \ref{fig:ch8:gm:perm}(A.I). We can apply a permutation matrix $P$ to swap the rows around with the following heuristic: If the matrix $P$ has an entry $p_{ji}$ which is one, then in the resulting matrix, the row $i$ will be the row $j$ from the matrix we permuted. 

For instance, in Figure \ref{fig:ch8:gm:perm}(A.II), the values $p_{12}$, $p_{23}$, $p_{34}$, and $p_{41}$ all have values of one, which means we will reorder the rows of $A$ so that $P^\top A$ will have the top row being the second row from the original matrix (and will have a value of two), the second row will be the third row from the original matrix (and will have a value of three), the third row will be the fourth row from the original matrix (and will have a value of four), and the fourth row will be the first row from the original matrix (and will have a value of one. 

We apply this ``row'' permutation with the matrix multiplication $P^\top A$:

\begin{lstlisting}[style=python]
import numpy as np

A = np.array([
    [1,1,1,1],
    [2,2,2,2],
    [3,3,3,3],
    [4,4,4,4]
])

P = np.array([
    [0,0,0,1],
    [1,0,0,0],
    [0,1,0,0],
    [0,0,1,0]
])

row_reordering = P.T @ A
\end{lstlisting}

We show a plot of the resulting row permutation in Figure \ref{fig:ch8:gm:perm}(A.III). Note that the row with a value of $1$ has been ``shifted'' to row $4$, and the values of rows $1$ through $3$ are $2$ through $4$ respectively. This is because of the way the rows/columns of the permutation matrix were arranged, as-per above.

\begin{figure}
    \centering
    \includegraphics[width=\linewidth]{applications/ch8/Images/gm_perm.png}
    \caption[Permutation matrices]{Row \textbf{(A)} shows a row permutation via pre-multiplication by $P^\top$, Row \textbf{(B)} shows a column permutation via post-multiplication by $P$, and row \textbf{(C)} shows a row and column permutation via pre-multiplication by $P^\top$ and post multiplication by $P$.}
    \label{fig:ch8:gm:perm}
\end{figure}
\subsubsection*{$BP$ moves the columns}

Likewise, a column permutation behaves very similarly. Let's now consider a matrix $B$, where the first column has a value of one, the second column has a value of two, the third column has a value of three, and the fourth column has a value of four, shown in Figure \ref{fig:ch8:gm:perm}(B.I). We use the same permutation matrix, where here, $p_{ji}$ indicates that column $i$ of the new matrix will be column $j$ from the matrix before the permutation was applied. This permutation is shown again in Figure \ref{fig:ch8:gm:perm}(B.II). We apply the column permutation matrix as $BP$:

\begin{lstlisting}[style=python]
B = np.array([
    [1,2,3,4],
    [1,2,3,4],
    [1,2,3,4],
    [1,2,3,4]
])

P = np.array([
    [0,0,0,1],
    [1,0,0,0],
    [0,1,0,0],
    [0,0,1,0]
])

column_reordering = B @ P
\end{lstlisting}

We show a plot of the resulting column permutation in Figure \ref{fig:ch8:gm:perm}(B.III). Note that the column with a value of $1$ has been ``shifted'' to column $4$, and the values of columns $1$ through $3$ are now $2$ through $4$ respectively. 

\subsubsection*{$P^\top DP$ moves the rows and columns concurrently}

As an interesting property of permutation matrices, we can apply these operations sequentially to reorder both the rows and columns of a matrix. Consider, for instance, a permutation matrix where row/column $1$ of the original matrix becomes row/column $2$ of the new matrix, and likewise, row/column $2$ of the original matrix becomes row/column $1$ of the new matrix. We'll consider a matrix $C$ where the first row and first column both have entries of all ones, and the rest of the matrix has the value zero. This matrix $C$ is shown in Figure \ref{fig:ch8:gm:perm}(C.I):

\begin{lstlisting}[style=python]
C = np.array([
    [1,1,1,1],
    [1,0,0,0],
    [1,0,0,0],
    [1,0,0,0]
])

P = np.array([
    [0,1,0,0],
    [1,0,0,0],
    [0,0,1,0],
    [0,0,0,1]
])

row_reordering = P.T @ C
row_column_reordering = row_reordering @ P
\end{lstlisting}

When we pre-multiply by $P^\top$, we end up ``swapping'' row $1$ with row $2$, as we would expect from what we learned in Figure \ref{fig:ch8:gm:perm}(A). When we post-multiply $P^\top C$ by $P$, we end up ``swapping'' columns $1$ and $2$, as we would expect from what we learned in Figure \ref{fig:ch8:gm:perm}(B). This results in jointly swapping the rows and columns $1$ and $2$, shown in Figure \ref{fig:ch8:gm:perm}(C.III). If $C$ were an adjacency matrix with nodes indexed $\{1, 2, 3, 4\}$, this would have the effect of permuting the node ordering of the adjacency matrix to $\{2, 1, 3, 4\}$.

\subsubsection*{Using concurrent row and column permutations on adjacency matrices}

For our networks, remember that the adjacency matrix is the matrix $A$ where the entry $a_{ij}$ represents whether or not there is an edge between nodes $i$ and $j$. The key aspect is that the indexing for the adjacency matrix, $ij$, is an indexing over a single set: the nodes. This means that if we want to reorder the adjacency matrix by moving around the nodes, we need to move both the rows and the columns concurrently, since the node ordering is what is being permuted. If we had a permutation of the nodes given by $P$, we would correspondingly reorder the adjacency matrix by permuting the rows and columns of $A$ by using $P^\top AP$.


\paragraph*{Permutation Matrices are unshuffled by their transpose}

Let's suppose that we have a permutation matrix, $P$. We remember from above that a permutation matrix is a matrix where every row and every column has a single entry which takes a value of one. Let's assume that $p_{ji} = 1$, which means that if we were to use the permutation as a row permutation, we would "flip" rows $i$ and $j$, or if we were to use it as a column permutation, we would "flip" columns $i$ and $j$. If we were to use it for both a row and column permutation, we would flip rows/columns $i$ and $j$. How do we ``undo'' this operation?

What happens when we take the product $P^\top P$? If for any pair of indices $p_{ji} = 1$, then $(P^\top)_{ij}=1$: the $(i, j)$ entry of the transpose is also one. This is just the definition of the matrix transpose operation. What does the matrix product of $P^\top$ and $P$ look like? Writing out the matrix multiplication, we see:
\begin{align*}
    P^\top P &= \begin{bmatrix}
    (P^\top)_{11} & ... & (P^\top)_{1n} \\
    \vdots & \ddots & \vdots \\
    (P^\top)_{n1} & ... & (P^\top)_{nn}
    \end{bmatrix}\begin{bmatrix}
    p_{11} & ... & p_{1n} \\
    \vdots & \ddots & \vdots \\
    p_{n1} & ... & p_{nn}
    \end{bmatrix}.
\end{align*}
When we use the definition of the transpose, this becomes:
\begin{align*}    P^\top P &= \begin{bmatrix}
    p_{11}& ... & p_{n1}\\
    \vdots & \ddots & \vdots \\
    p_{1n} & ... & p_{nn}
    \end{bmatrix}\begin{bmatrix}
    p_{11} & ... & p_{1n} \\
    \vdots & \ddots & \vdots \\
    p_{n1} & ... & p_{nn}
    \end{bmatrix}.
\end{align*}
The resulting matrix $P^\top P$ has entries $i, j$ where:
\begin{align*}
(P^\top P)_{ij} = \sum_{k = 1}^n p_{ik}p_{jk}
\end{align*}
But, as we know, for a particular row $i$ and column $k$, exactly a single entry has a value of $1$. This means that for any $i \neq j$,  $p_{ik}p_{jk}$ will always be equal to zero, because you could not have two rows of the same column $k$ both taking the value of $1$ concurrently.

If $i = j$, then there must be some $k$ where $p_{ik} = 1$, because at least $1$ entry of the columns of $P$ must be $1$ by definition.

Therefore, $(P^\top P)_{ij} = 1$ if $i = j$, and $(P^\top P)_{ij} = 0$ everywhere else. This is the definition of the identity matrix, so $P^\top P = I$. Since the transpose of the identity matrix is also the identity matrix, $PP^\top = I$, too.

This has the interpretation that if we permute an adjacency matrix's rows and columns $A$ with a permutation matrix $P$, giving us $B = P^\top A P$, that we can "undo" this permutation by taking $PBP^\top$. We can see this by just looking at it, and plugging in the definition of $B$:
\begin{align*}
    PBP^\top &= P\left(P^\top A P\right)P^\top, \\
    &= PP^\top A PP^\top, \\
    &= I A I = A,
\end{align*}
where  used the fact that $PP^\top = I$.

In this sense, if we think of $P^\top A P$ being the row/column permuted adjacency matrix of $A$, then permuting it again with $P^\top$ instead of $P$ will undo the permutation.

\begin{floatingbox}[h]\caption{Concept: Unshuffling a shuffled adjacency matrix}
\label{box:ch8:gm:unshuffle}
Suppose that $B$ is a shuffling of the adjacency matrix $A$ by $P$; that is, $B = P^\top AP$. Then the $B$ can be unshuffled by permuting $B$ with the matrix $P_u$, where $P_u = P^\top$, and:
\begin{align*}
    P_u^\top B P_u &= P B P^\top \\
     &= PP^\top A P^\top P \\
     &= A,
\end{align*}
because $PP^\top = P^\top P = I$.
\end{floatingbox}

\subsubsection{Permutation Matrices to Match Networks}

Let's go back to Twitter and Facebook. Remember that we had two networks, where there was a node correspondence in that person $0$ from Twitter was the same as person $a$ from Facebook, person $1$ from Twitter was the same as the person $b$ from Facebook, so on and so forth. 

We will assume that the nodes from Twitter are given to us in order, $\{0, 1, 2, 3\}$. In the ideal case, the nodes from Facebook will respect the node correspondence and be ordered as $\{a, b, c, d\}$. The problem we illustrated in Equation \eqref{eqn:ch8:gm:fb_perm:e1} was that, if the nodes for Facebook were ordered $\{a, b, d, c\}$, then $f\left(A^{(T)}, A^{(F'')}\right) = 8$.

\begin{figure}[h]
    \centering
    \includegraphics[width=\linewidth]{applications/ch8/Images/gm_fb_tw.png}
    \caption[Facebook and Twitter example, with permutations applied]{\textbf{(A)} The adjacency matrix for Twitter $A^{(T)}$, \textbf{(B)} The adjacency matrix for Facebook, $A^{(F)}''$, \textbf{(C)} The adjacency matrix for Facebook, after swapping nodes $c$ and $d$ via a permutation matrix.}
    \label{fig:ch8:gm:fb_tw}
\end{figure}

We want to construct a permutation matrix $P$, which will keep the nodes $a$ and $b$ in the same order, but swap nodes $c$ and $d$ in the node ordering. We can do this using the strategy that we developed above:

\begin{lstlisting}[style=python]
twitter = np.array([
    [0,1,1,0],
    [1,0,0,1],
    [1,0,0,1],
    [0,1,1,0]
])

facebook = np.array([
    [0,1,0,1],
    [1,0,1,0],
    [0,1,0,1],
    [1,0,1,0]
])

P = np.array([
    [1,0,0,0],
    [0,1,0,0],
    [0,0,0,1],
    [0,0,1,0]
])

fb_permutation = P.T @ facebook @ P
\end{lstlisting}
The Facebook adjacency matrix (after reshuffling) is shown in \ref{fig:ch8:gm:fb_tw}(C). Note that the permutation matrix swaps nodes $c$ and $d$, to recover the original node correspondence between Twitter and Facebook, and the networks are identical after pre- and post-multiplying by the permutation matrix.

\subsubsection*{Formalizing the Graph Matching Problem}

We will use this intuition to formulate the graph matching problem. For any two adjacency matrices $A, B$, we seek to minimize the cost function $g_P(A,B) = \left|| A - P^\top BP\right\|_F^2$ with the restriction that $P$ is a permuation matrix. This means that you want to figure out a way in which you can shuffle the rows and columns of $B$, such that it is as close as possible to $A$.
\subsubsection*{Generating a random permutation matrix}

Let's make ourselves a function which creates permutation matrices. Remember that for a permutation matrix, the entry $p_{ji}$ corresponds to a swap of rows/columns $i$ and $j$, depending on whether it is a row or column permutation (or being used for both).

\begin{lstlisting}[style=python]
def make_permutation(n):
    """
    A function that generates a permutation for n elements.
    
    1. Generate indices from 0 to n-1
    2. shuffle those indices
    3. Place 1s in the matrix P at the positions defined by the shuffled indices.
    """
    
    starting_indices = np.arange(n)
    destination_indices = np.random.permutation(n)
    P = np.zeros(shape=(n,n))
    P[destination_indices, starting_indices] = 1
    return P
\end{lstlisting}

\subsection{Finding a good permutation with gradient descent optimization}

To solve the optimization problem we described above, we'll use a variation of gradient descent. A gradient can be thought of as a vector valued slope; it is simply the slope of a function in all of its dimensions, at a single point in space. Gradient Descent is a common optimization method used to find minimums of functions.

\begin{figure}[h]
    \centering
    \includegraphics[width=\linewidth]{applications/ch8/Images/grad_desc.png}
    \caption[Gradient descent]{\textbf{(A)} the cost function with an initial starting position, \textbf{(B)} the cost function, optimized via gradient descent.}
    \label{fig:ch8:gm:grad_desc}
\end{figure}

The graph matching optimization problem, unfortunately, requires a background in nonlinear optimization, which is beyond the scope of this textbook. For that reason, we will focus heavily on the types of problems graph matching can solve in this chapter. If you do have such a background, please see the references for a more algorithmic understanding. For now, you can just think of the problem as being solved with something similar to gradient descent.

\begin{floatingbox}[h]\caption{Gradient Descent}

You can think of gradient descent like gravity. Consider an inspector using a golf ball to find the lowest point when installing a drain. The ball rolls down hill until it comes to a stop; once it stops, we know we've found the lowest point. Gradient descent works in a similar way, taking steps in the direction of the local gradient with respect to some parameter. Once the gradient is zero, a local minimum has been found and the algorithm is stopped. 

This process is illustrated in Figure \ref{fig:ch8:gm:grad_desc}, for a one-dimensional parameter $\theta$. The $y$-axis represents the cost $f_\theta(X)$ of a particular parameter choice $\theta$, given the data $X$ (solid line). In Figure \ref{fig:ch8:gm:grad_desc}(A), an initial parameter value $\theta_0$ is chosen to begin the optimization routine. The gradient is computed at the point $\theta_0$, which with a one-dimensional parameter, is the slope of the tangent line to $f_\theta(X)$ at $\theta_0$. Since the slope is negative, this indicates that increasing the value of the parameter an arbitrarily small amount past $\theta_0$ will decrease the cost. If the slope were positive, decreasing the value of the parameter an arbitrarily small amount past $\theta_0$ would decrease the cost. Figure \ref{fig:ch8:gm:grad_desc}(B) illustrates the effects of repeating this process. Successive ``learning steps'' repeat this process until the tangent line has a slope of zero, at which point a local minimum for the cost function $f_\theta(X)$ at $\theta^*$ has been found. 

The main steps of a gradient descent method are choosing a suitable initial position (can be chosen randomly), then gradually improving the cost function one step at a time, until the function is changing by a very small amount, converging to a minimum. The main issue with gradient descent is that it does not guarantee that you will find a global minimum, only that you will find a local minimum to your initial position (as long as your function is sufficiently smooth).  A commonly used strategy, and the one that we employ for graph matching, is known as the Fast Approximate Quadratic (FAQ) algorithm \cite{Vogelstein2015Apr}.

\end{floatingbox}

\subsection{Solving the graph matching problem}

For the example below, we will match two networks with a known node mapping that preserves a common network structure. To do this, we simulate a single sample from a $ER_8(0.5)$ random network. Then, we generate $B$ by randomly permuting the node labels of $A$. 

\begin{lstlisting}[style=python]
from graspologic.simulations import er_np

n = 8
p = 0.5

np.random.seed(1234)
A = er_np(n=n, p=p)
# make a permutation matrix
P = make_permutation(n)
B = P.T @ A @ P
disagreements = np.linalg.norm(A - B)**2
\end{lstlisting}

The network sample is shown in Figure \ref{fig:ch8:gm:simp_ex}(A), and the network after a random shuffling of the nodes is shown in Figure \ref{fig:ch8:gm:simp_ex}(B). 

\begin{figure}
    \centering
    \includegraphics[width=\linewidth]{applications/ch8/Images/gm_simp_ex.png}
    \caption[Unshuffling to solve the graph matching problem]{\textbf{(A)} the original adjacency matrix, \textbf{(B)} the shuffled adjacency matrix $B$, using the random permutation $P$, \textbf{(C)} the unshuffling of $B$, after graph matching to $A$, \textbf{(D)} the edge disagreement matrix.}
    \label{fig:ch8:gm:simp_ex}
\end{figure}

Below, we create a model to solve the Graph Matching Problem using the \texttt{graph\_match} function. We pass in two networks, \texttt{A} and \texttt{B}, that we wish to ``match.''

\begin{lstlisting}[style=python]
from graspologic.match import graph_match

gmp = graph_match(A,B)
\end{lstlisting}

The \texttt{graph\_match} function returns a \texttt{MatchResult} object, which in its most straightforward application (the networks \texttt{A} and \texttt{B} have the same number of nodes), will return two indexing sets, \texttt{indices\_A} and \texttt{indices\_B}. Each element of these indexing sets correspond to the ``matches'' of nodes; that is, \texttt{indices\_A[j]} is ``matched'' to node \texttt{indices\_B}.

In this simple setting, \texttt{indices\_A} will just be the nodes ordered from $0$ to $n-1$, and indices \texttt{indices\_B} will be a map from the (current) indices of nodes in \texttt{B} that would align them to nodes in \texttt{A}.

We can use these indices to construct an ``unshuffling'' permutation matrix for network \texttt{B}, using the logic that we developed above for permutation matrices. We want a matrix $P_u = P^\top = P^{-1}$ where each element $p_{ij}$ is $1$ if $j = \texttt{indices\_B[i]}$, and $0$ otherwise. The original matrix, $P$, is the matrix that permuted $A$ to create $B$ in the first place.

\begin{lstlisting}[style=python]
def make_unshuffler(destination_indices):
    """
    A function which creates a permutation matrix P from a given permutation of the nodes.
    """
    n = len(destination_indices)
    P_u = np.zeros((n, n))
    starting_indices = np.arange(n)
    P_u[destination_indices, starting_indices] = 1
    return P_u

P_u = make_unshuffler(gmp.indices_B)
B_unshuffled = P_u.T @ B @ P_u
disagreements = np.linalg.norm(A - B_unshuffled)**2
\end{lstlisting}

In this case, we are estimating the unshuffling  matrix, so we produced an estimate $\hat P_u$. When we unshuffle $B$ with $\hat P_u$, we obtain the matrix $\hat P_u^\top B \hat P_u$, which is shown in Figure \ref{fig:ch8:gm:simp_ex}(C). Note that there are no edge disagreements between $\hat P^\top B \hat P$ and $A$, shown in Figure \ref{fig:ch8:gm:simp_ex}(D). 

The algorithm used is randomly initialized, so when you run this algorithm on your computer, you might not get a perfect unshuffling (but they should generally be pretty similar to what we got). 

\subsubsection*{The match ratio of nodes}

We can evaluate the quality of an unshuffling using the ratio of nodes which are correctly matched, called the \textit{match ratio}. We will use our knowledge of how $P$ behaves from Concept \ref{box:ch8:gm:unshuffle} to calculate this metric.

Because permutation matrices are orthogonal, $PP^\top = PP^{-1}$ is the identity matrix $I$. The match ratio is just the extent to which $P_u$ is the inverse of $P$, which is equivalent to the extent to which $P_u P$ is the identity matrix.

More specifically, every node which is incorrectly matched will correspond to an entry of $0$ along the diagonal of $PP_u$ (or equivalently, of $P_u P$).

With this in mind, we can just take the match ratio to be the fraction of times the diagonal of $P_uP$ of $PP_u$ is $1$:
\begin{align*}
    \text{match ratio}(P, P_u) = \frac{1}{n} \sum_{i = 1}^n \mathds 1\left\{(PP_u)_{ii} = 1\right\}
\end{align*}
where $P$ is a permutation matrix and $P_u$ is a proposed unshuffling matrix. The fancy looking function $\mathds 1\{x\}$ is just an indicator that has a value of $1$ if the statement inside the inside the curly braces is true, and $0$ if the statement inside the curly braces is false. So, here, it just has a value of $1$ if $(PP_u^\top)_{ii} = 1$, and a value of $0$ if $(PP_u^\top)_{ii} \neq 1$. We write a simple utility to do this, and then can call it on our permutation and un-shuffling matrix to see that the match ratio is $1$ here (we perfectly unshuffled $B$):

\begin{lstlisting}[style=python]
def match_ratio(P, Pu):
    n = len(P) # the number of nodes
    diag = np.diag(P @ Pu)
    return (diag == 1).sum() / n

print("match ratio: {:.3f}".format(match_ratio(P, P_unshuffle)))
# match ratio: 1.000
\end{lstlisting}

\subsection{Seeded graph matching (SGM) on correlated network pairs}

As networks become larger, they quickly become more difficult to match. One method to mitigate this difficulty is to use \textit{seeds}. \textit{Seeds} are a subset of matches that we already know before we perform the graph matching. For example, if we are given two networks $T$ and $F$ with 300 nodes each, we might already know ten node matches between $T$ and $F$. Having this prior information dramatically improves our ability to match the networks. 

To demonstrate the effectiveness of Seeded Graph Matching (\texttt{SGM}) \cite{Fishkind2019Mar, Lyzinski2014Jan}, the algorithm will be applied on a pair of simpler correlated SBM networks, which is an adaptation of the $\rho$-correlated $RDPG$ which we learned about in Section \ref{sec:ch5:multi:corr}. Like the $\rho$-correlated $RDPG$, the idea here is that we have two normal SBMs, but for any edge in the two networks $\mathbf a_{ij}$ and $\mathbf b_{ij}$, they will be correlated with correlation $\rho$. If there is sufficient correlation in edge structure between these two networks, we can hope to align the nodes of the two networks on the basis of this edge structure. The block matrix is:
\begin{align*}
B &= \begin{bmatrix} 
0.7 & 0.3 & 0.4\\
0.3 & 0.7 & 0.3\\
0.4 & 0.3 & 0.7
\end{bmatrix}
\end{align*}
The first $75$ nodes in the network will be from community one, the second $75$ nodes in the network will be from community two, and the third $75$ nodes in the network will be from community three:

\begin{lstlisting}[style=python]
from graspologic.simulations import sbm_corr

n_per_block = 75
n_blocks = 3
block_members = np.array(n_blocks * [n_per_block])
n_nodes = block_members.sum()
rho = 0.9
block_probs = np.array(
    [[0.7, 0.1, 0.4], 
     [0.1, 0.3, 0.1], 
     [0.4, 0.1, 0.7]]
)

A, B = sbm_corr(block_members, block_probs, rho)
disagreements = np.linalg.norm(A - B)**2
\end{lstlisting}

The networks $A$ and $B$ are shown in Figure \ref{fig:ch8:gm:sgm:ex_nets}(A) and (B). Note that the networks have a similar topological structure. 
\begin{figure}[h]
    \centering
    \includegraphics[width=\linewidth]{applications/ch8/Images/gm_sgm_nets.png}
    \caption[Seeded graph matching, networks (unshuffled)]{\textbf{(A)} The network $A$, \textbf{(B)} the network $B$ which is $\rho$-correlated to $A$, \textbf{(C)} the network $B^{(r)}$, which is the network $B$ with $35$ nodes removed per community.}
    \label{fig:ch8:gm:sgm:ex_nets}
\end{figure}

To emphasize the effectiveness of \texttt{SGM}, as well as why having seeds is important, we will randomly shuffle the vertices of network $B$. 

\begin{lstlisting}[style=python]
P = make_permutation(n_nodes)
B_shuffle = P.T @ B @ P
disagreements_shuffled = np.linalg.norm(A - B_shuffle)**2
print("Number of adjacency disagreements: {:d}".format(int(disagreements_shuffled)))
\end{lstlisting}

We will call this version of $B$ after shuffling $B^{(s)}$. The network $A$ and the shuffled network $B^{(s)}$ are shown in Figure \ref{fig:ch8:gm:sgm:ex}, along with the edge disagreements.

\begin{figure}[h]
    \centering
    \includegraphics[width=\linewidth]{applications/ch8/Images/gm_seed_ex.png}
    \caption[Seeded graph matching example]{\textbf{(A)} the original network $A$, \textbf{(B)} the shuffled network $B^{(s)}$ which is $\rho$-correlated to $A$, \textbf{(C)} the edge disagreements between $A$ and $B^{(s)}$.}
    \label{fig:ch8:gm:sgm:ex}
\end{figure}

\paragraph*{Matching the networks without seeds}

First, we will run SGM on network $A$ and the shuffled network $B_s$ with no seeds, using a similar approach to the above


\begin{lstlisting}[style=python]
# fit with A and shuffled B
sgm = graph_match(A, B_shuffle)

# obtain unshuffled version of the shuffled B
P_unshuffle_noseed = make_unshuffler(sgm.indices_B)
B_unshuffle_noseed = P_unshuffle_noseed.T @ B_shuffle @ P_unshuffle_noseed

# compute the match ratio
n_verts = len(P)
match_ratio_noseed = np.count_nonzero(np.diag(P_unshuffle_noseed.T @ P))/n_verts
disagreements_noseed = np.linalg.norm(A - B_unshuffle_noseed)**2

print("Match Ratio, no seeds: {:.3f}".format(match_ratio_noseed))
print("Disagreements, no seeds: {:d}".format(int(disagreements_noseed)))
\end{lstlisting}

While the predicted unshuffling for $B$ was relatively successful in recovering the basic structure of the network $A$, we see that the number of edge disagreements between them is still quite high, and the match ratio of successfully unshuffled nodes is quite low (almost $0$). The original network $A$, the unshuffled network $\hat P_u^\top B^{(s)}\hat P_u$, and the edge disagreements between the original network and the unshuffled correlated network are shown in Figure \ref{fig:ch8:gm:sgm:noseed}. Note that the edge disagreements are fairly frequent.

\begin{figure}[h]
    \centering
    \includegraphics[width=\linewidth]{applications/ch8/Images/gm_sgm_noseed.png}
    \caption[Graph matching with no seeds]{\textbf{(A)} the original network $A$, \textbf{(B)} the unshuffling of the correlated network $\hat P_u^\top B^{(s)}\hat P_u$, \textbf{(C)} the edge disagreements between the original network and the unshuffled network.}
    \label{fig:ch8:gm:sgm:noseed}
\end{figure}

\paragraph*{Matching the networks with seeds}

Next, we will run SGM with 10 seeds randomly selected from the optimal permutation vector found ealier. Although 10 seeds is only about 3\% of the 300 node network, we will observe below how much more accurate the matching will be compared to having no seeds. We add a little helper function, which takes a permutation matrix and a desired number of seeds, and indicates where the identified seed nodes would be permuted to:

\begin{lstlisting}[style=python]
def gen_seeds(P, n_seeds):
    """
    A function to generate n_seeds seeds for a pair of matrices A and P^TBP
    which are initially matched, but P has been applied to permute the nodes
    of B.
    """
    n = P.shape[0]
    # obtain n_seeds random seeds from 1:n
    seeds = np.random.choice(n, size=n_seeds, replace=False)
    # use the permutation matrix to find where each seed was permuted to
    seeds_permuted = [np.where(P[i, :] == 1)[0] for i in seeds]
    return np.hstack(
        (seeds.reshape(n_seeds, 1), 
         np.array(seeds_permuted).reshape(n_seeds, 1))
    )
\end{lstlisting}

Next, we run seeded graph matching, using the \texttt{graph\_match} function from \texttt{graspologic}, by passing seeds as parameters:
\begin{lstlisting}[style=python]
nseeds = 10  # the number of seeds to use
# select ten nodes at random from A which will serve as seeds

# obtain seeds for nodes of A with nodes of B
seeds = gen_seeds(P, nseeds)

# run SGM with A and shuffled B, but provide the seed nodes from A as ref_seeds
# and the corresponding position of these seed nodes after shuffling as permuted_seeds
sgm = graph_match(A, B_shuffle, partial_match=seeds)
P_unshuffle_seeds = make_unshuffler(sgm.indices_B)

B_unshuffle_seeds = P_unshuffle_seeds.T @ B_shuffle @ P_unshuffle_seeds

match_ratio_seeds = match_ratio(P, P_unshuffle_seeds)
disagreements_seeds = np.linalg.norm(A - B_unshuffle_seeds)**2

print("Match Ratio, seeds: {:.3f}".format(match_ratio_seeds))
# Match Ratio with seeds: 1.000
print("Disagreements, seeds: {:d}".format(int(disagreements_seeds)))
\end{lstlisting}

The resulting unshuffling steps are shown in Figure \ref{fig:ch8:gm:sgm:seed}. Compared to Figure \ref{fig:ch8:gm:sgm:noseed}, we can see that the unshuffling produces far fewer edge disagreements than when we used unseeded graph matching. Further, using just $10$ seeds, the match ratio increased to at or near perfect (it should be near $1$). 

\begin{figure}[h]
    \centering
    \includegraphics[width=\linewidth]{applications/ch8/Images/gm_sgm_seed.png}
    \caption[Seeded graph matching]{\textbf{(A)} the original matrix $A$, \textbf{(B)} the unshuffling of the correlated matrix $\hat P_u^\top B^{(s)}\hat P_u$ and \textbf{(C)} the edge disagreements when we graph match using seeds.}
    \label{fig:ch8:gm:sgm:seed}
\end{figure}

So, our conclusions can be summarized as follows:
\begin{enumerate}
    \item When one network is ``mis-aligned'' to another network, in that the nodes are not ordered the same but the network is otherwise identical, graph matching strategies can efficiently recover an unshuffling matrix to ``align'' the nodes between the two networks.
    \item When two networks have ``mis-aligned'' nodes and further are only correlated (rather than identical, up to the alignment of the nodes), graph matching strategies will, naively, struggle to find optimal solutions. However, if we can narrow down the scope of the problem via seeds, we can still yield precise solutions.
\end{enumerate}

\subsection{Padded graph matching}

From what we've seen so far, the two networks you are interested in matching nodes for must have the same number of nodes. In practice, this is a pretty restrictive limitation. You might come across pairs of networks where many, if not all, of the nodes in a smaller network are matched to a node in a larger network. In this case, you have a dilemma: how do you match the nodes between the networks, without knowing what to do with the extra nodes in the larger network?

We will do this through a technique called \textit{padded graph matching}, in which we add new nodes to the smaller network until it has the same number of nodes as the bigger network, and then we run graph matching on the resulting networks with an equal number of nodes.  We have two techniques to add these isolated nodes, naive and adaptive padding.

For these examples, we'll adjust our correlated network slightly. We'll keep our first network $A$ exactly like the network we sampled above from the $\rho$-SBM. For $B$, we'll arbitrarily take out the last 35 nodes of each block:

\begin{lstlisting}[style=python]
from graspologic.utils import remove_vertices
import numpy as np

nremove = 25

# nodes to remove from A
n_nodes_Brem = n_nodes - nremove*n_blocks
base_range = np.arange(n_per_block - nremove, n_per_block)
block_offsets = np.array([0, 75, 150])

# repeat a base range for each block and add block offsets
nodes_to_remove = np.repeat(base_range, len(block_offsets)) 
nodes_to_remove += np.tile(block_offsets, nremove)
nodes_to_retain = np.setdiff1d(np.arange(n_nodes), nodes_to_remove)

# use the remove_vertices function to remove nodes
B_rem = remove_vertices(B, nodes_to_remove)
\end{lstlisting}

In the above code, note the care taken to obtain node indices of the nodes from $B$ that are retained in $B^{(r)}$. This is so that we will be able to evaluate our graph matching after we apply graph matching techniques. The network with the last $35$ nodes removed from network $B$ is shown in Figure \ref{fig:ch8:gm:sgm:ex_nets}(C).

This leaves us with a network $B^{(r)}$ and corresponding underlying random network $\mathbf B^{(r)}$ in which there are only $150$ instead of $225$ nodes. These $150$ nodes are matched to $150$ of the $225$ nodes in $A$ and $\mathbf A$, respectively. We won't shuffle $B^{(r)}$ this time for visualization purposes, but the procedure below is identical when the network is shuffled.

Our task is to match the $150$ nodes in $B^{(r)}$ to their corresponding matched pair in $A$.

Behind the scenes, what we want to do is basically take the network $A$, and match $B^{(r)}$ to a subnetwork of $A$ induced by the nodes for which there is a corresponding matched pair. By this, what we mean is that we want to figure out which nodes in the larger network $A$ actually have a matched pair in $B^{(r)}$, and virtually ignore the other nodes entirely. In this case, the induced subnetwork of $A$ can be obtained like this:

\begin{lstlisting}[style=python]
A_induced = remove_vertices(A, nodes_to_remove)
\end{lstlisting}

which is exactly equivalent to:

\begin{lstlisting}[style=python]
A_induced = A[nodes_to_retain,:][:,nodes_to_retain]
\end{lstlisting}

\subsubsection*{Naive padded graph matching}

Through naive padding, we simply add isolated nodes to the smaller network (which is $B$, in your case), until the number of nodes in $B$ are equal to the number of nodes in $A$. If you remember from Section \ref{sec:ch4:regularization}, nodes in a simple network are just nodes which do not have any edges in the network) The padded version of $B$ can be obtained like this:

\begin{lstlisting}[style=python]
B_padded = np.pad(
    B_rem, 
    pad_width=[(0,nremove*n_blocks), (0, nremove*n_blocks)]
)
\end{lstlisting}

which makes the number of nodes in the two networks the same. The padded network is shown in Figure \ref{fig:ch8:gm:sgm:naive_padded}(B).

We can specify this by using the \texttt{graph\_match} function, using the argument \texttt{partial\_match=seeds}. Then, we re-run graph matching, optionally, using seeding, just like we did before. This time, after padding, the original seeds of the network should come from only the ``retained'' nodes; that is, we want to make sure our seeds actually come from the smaller network nodes that are actually part of network \texttt{B\_rem}, and are not the padded nodes that we added:

\begin{lstlisting}[style=python]
nseeds_padded = 5

# obtain which nodes of B will be the seeds to use, from the retained
# nodes in the network
rem_seeds = np.random.choice(n_nodes_Brem, 
              size=nseeds_padded, replace=False).reshape(-1, 1)

# obtain the nodes in A
ref_seeds = nodes_to_retain[rem_seeds.reshape(-1)].reshape(-1, 1)
seeds = np.hstack((ref_seeds, rem_seeds))

# run SGM with A and the padded network B
# since we didn't shuffle Br, the seeds are the same for both
gmp_naive = graph_match(A, B_padded, partial_match=seeds)

# unshuffle B using the padded version of B and the permutation identified
P_unshuffle = make_unshuffler(gmp_naive.indices_B)
B_unshuffle_seeds_naive = P_unshuffle.T @ B_padded @ P_unshuffle


\end{lstlisting}

The network after matching is shown in Figure \ref{fig:ch8:gm:sgm:naive_padded}(C). 

Unfortunately, The naive matching of $B^{(r)}$ after padding looks nothing like $A$ we were trying to match it to; particularly, notice that all of the ``padding nodes'' (with no edges) were simply ``matched'' with the second community of nodes. Notice that there is a large ``hole'' in the matched network in the middle, with very few edges. What happened?

\begin{figure}
    \centering
    \includegraphics[width=\linewidth]{applications/ch8/Images/gm_naive.png}
    \caption[Naive padded graph matching]{\textbf{(A)} the original network $A$, \textbf{(B)}the network $B$ is $\rho$-correlated with $A$, has $35$ nodes removed to form $B^{(r)}$, and then has $35$ isolates added to pad $B^{(r,p)}$ to create the padded network, \textbf{(C)} the padded network after graph matching.}
    \label{fig:ch8:gm:sgm:naive_padded}
\end{figure}

\paragraph*{Naive matching matches padded nodes to low density subnetworks}

When we used this naive approach for padded graph matching, we made a critical error. We took the isolated nodes of $B$ that we added (just to make the number of nodes align) and attempted to, in some sense, consider these nodes as ``equals'' to the remaining nodes in the network in our matching. These nodes ended up being aligned to low-density subnetworks of $A$, which means that we allowed nodes that didn't really exist in $B^{(r)}$ (the padded nodes) to have a substantial role in the match quality. 

Looking at \ref{fig:ch8:gm:sgm:naive_padded}(C), let's consider the model that our networks were generated with. The average degree for nodes in community two are lower than the average degree for nodes in communities one and three, which means that these nodes will (in general) comprise the lowest density subnetwork of $A$. 

When we perform our matching, the isolated nodes from $B^{(r, p)}$ were ``aligned'' to the nodes in community two, which we can see in \ref{fig:ch8:gm:sgm:naive_padded}(C) by noting that the middle of the naive matched network is largely comprised of the nodes that we added synthetically with no edges (the isolates). Remember that these nodes are basically just placeholders, so we do not want them to have such a deciding role on our network.

To perform naive matching in \texttt{graspologic}, we can use the padding argument with the network \texttt{B\_rem} directly. This will match the nodes of the smaller of the two networks to the other, and the \texttt{indices\_*} return argument will indicate the nodes of each network that are matched to non-padding nodes:

\begin{lstlisting}[style=python]
# run naive padded SGM with A and the smaller network B_rem
# specifying padding="naive" internally does the same thing as we did above
gmp_naive = graph_match(A, B_rem, partial_match=seeds, padding="naive")

# unshuffle B using the network B with nodes removed
# and the permutation identified
P_unshuffle = make_unshuffler(gmp_naive.indices_B)
Brem_unshuffle_seeds_naive = P_unshuffle.T @ B_rem @ P_unshuffle
\end{lstlisting}

We can use \texttt{indices\_B} to evaluate the matching ratio and the number of agreements/disagreements, like we did before. If we recover the match perfectly, the unshuffling permutation matrix will be the identity matrix, since we did not reorder the nodes of $B^{(r)}$ when we formed the array \texttt{nodes\_to\_retain}:

\begin{lstlisting}[style=python]
# evaluate the match ratio by computing the permutation of the nodes of A,
# which should just be the identity matrix
match_ratio_naive = match_ratio(np.eye(len(nodes_to_retain)), P_unshuffle)
disagreements_naive = np.linalg.norm(A_induced - Brem_unshuffle_seeds_naive)**2
print("Match Ratio, naive padding: {:.3f}".format(match_ratio_naive))
# Match Ratio, naive padding: 0.333
print("Disagreements, naive padding: {:d}".format(int(disagreements_naive)))
\end{lstlisting}

Which gives us a low match ratio and a relatively high number of disagreements. As an exercise, plot \texttt{A\_induced}, and compare it to \texttt{B\_unshuffle\_seeds\_naive}. You should see that the networks look nothing alike, which means that our ``matching'' hasn't really done anything of value for us.

\subsubsection*{Adopted Padded Graph Matching}

Instead, what we want to do is match $B^{(r)}$ to the best fitting induced subnetwork of $A$. The key difference is that, in the ideal case, the subnetwork induced on $A$ is the set of nodes which were actually retained by $B$, and not just any ordinary subnetwork (or, in the case of naive padding, the lowest density subnetwork being matched to padding nodes).

To do this, we use a strategy called adopted padding, which is performed using `padding="adopted"` for the `GraphMatch` object. Through adopted padding, we normalize $A$ and $B^{(r)}$, to form $\tilde A$ and $\tilde B$, by multiplying the networks by 2, and then subtracting a matrix of $1$s. We again pad $\tilde B$ exactly like we did before, adding isolated nodes until the two networks have the same number of nodes. Through this normalization scheme, the padded nodes (which are not actual nodes in $B^{(r)}$) are \textit{downweighted}, such that they will tend to matter less in the cost function. What this has the effect of is making it so we will discount the padded nodes entirely when performing our graph matching, and will allow us to find the best induced subnetwork (on the nodes that are actually in the network $B^{(r)}$) instead. We perform this as follows:

\begin{lstlisting}[style=python]
Atilde = 2 * A - np.ones(A.shape[0])
Btilde = 2*B_rem - np.ones(B_rem.shape[0])
Btilde_padded = np.pad(Btilde, [(0,nremove*n_blocks), (0, nremove*n_blocks)])

# run SGM with Atilde and the padded network Btilde
# since we didn't shuffle Br, the seeds are the same for both
gmp_adopted = graph_match(Atilde, Btilde_padded, partial_match=seeds)

# unshuffle B using the padded version of B and the permutation identified
P_unshuffle = make_unshuffler(gmp_adopted.indices_B)
B_unshuffle_seeds_adopted = P_unshuffle.T @ B_padded @ P_unshuffle
\end{lstlisting}

The adopted padded and matched networks \texttt{A} and \texttt{B\_unshuffle\_seeds\_adopted} are shown in Figure \ref{fig:ch8:gm:sgm:adopted}(A) and Figure \ref{fig:ch8:gm:sgm:adopted}(B) respectively. Note that the isolated nodes tend to be dispersed to the last few nodes of each community, which is consistent with the nodes that were originally removed from the network. They are not restricted to the nodes in the lowest density induced subnetwork, as in Figure \ref{fig:ch8:gm:sgm:naive_padded}.

To evaluate this padding scheme, we can look evaluate the matching ratio and the on the subnetwork induced by the retained nodes. We can again do this by using \texttt{graspologic}, which will perform the above operation, and then induce the subnetwork of nodes retained by both networks (implied by the matching to non-padded nodes) automatically, like above:

\begin{lstlisting}[style=python]
# run SGM with A and Br with nodes removed
# since we didn't shuffle Br, the seeds are the same for both
gmp_adopted = graph_match(A, B_rem, partial_match=seeds, padding="adopted")

# unshuffle B using the padded version of B and the permutation identified
P_unshuffle = make_unshuffler(gmp_adopted.indices_B)
Brem_unshuffle_seeds_adopted = P_unshuffle.T @ B_rem @ P_unshuffle

match_ratio_adopted = match_ratio(np.eye(len(nodes_to_retain)), P_unshuffle)
disagreements_adopted = np.linalg.norm(A_induced - Brem_unshuffle_seeds_adopted)**2
print("Match Ratio, adopted padding: {:.3f}".format(match_ratio_adopted))
# Match Ratio, adopted padding: 1.000
print("Disagreements, adopted padding: {:d}".format(int(disagreements_adopted)))
\end{lstlisting}

Using adopted matching has increased the match ratio to perfect, and the number of disagreements has been dramatically reduced.

The network \texttt{A\_induced} induced by the nodes included is shown in Figure \ref{fig:ch8:gm:sgm:adopted}(C), alongside the network \texttt{Brem\_unshuffle\_seeds\_adopted} with adopted matching in Figure \ref{fig:ch8:gm:sgm:adopted}(D). This appears much more reasonable than the outcome we saw in Figure \ref{fig:ch8:gm:sgm:naive_padded}.

\begin{figure}
    \centering
    \includegraphics[width=\linewidth]{applications/ch8/Images/gm_adopted.png}
    \caption{\textbf{(A)} The network $A$, \textbf{(B)} the padded correlated network $B^{(r,p)}$ after row/column permutation by $\hat P_u$ via adopted matching, \textbf{(C)} the subnetwork of $A$ induced by the nodes in $B^{(r)}$, \textbf{(D)} the correlated network $B^{(r)}$ after row/column permutation by $\hat P_u$.}
    \label{fig:ch8:gm:sgm:adopted}
\end{figure}

\begin{floatingbox}[h]\caption{Remark: Vertex Nomination via Seeded Graph Matching}
It is often the case that we might have two networks $A^{(1)}$ and $A^{(2)}$, and we might want to ask which node or set of nodes in a second network are ``maximally similar'' to a given node (or set of nodes) of interest in the first network. Let's call this node $i^{(1)}$. This can be conceptualized as a form of the vertex nomination problem from Section \ref{sec:ch7:vn}, where we have a node of interest (or set of nodes of interest) and want to produce a \textit{nomination list} of possible nodes in the network $A^{(2)}$ that our node(s) of interest in the network $A^{(1)}$ are most similar to. 

Given that the solution to the graph matching problem is non-deterministic, in that running it twice might produce a slightly different matching (because we use stochastic gradient descent), we can estimate a nomination list by computing the number of times each node $j^{(2)}$ in network $2$ are matched to node $i^{(1)}$. Then, we can order the nomination list by the nodes that are most frequently matched to node $i^{(1)}$. This procedure is known as \texttt{VNviaSGM} \cite{Patsolic2020Jun}.
\end{floatingbox}

\newpage

\section{Vertex nomination via seeded graph matching}
\label{sec:ch8:vnviasgm}



\bibliographystyle{vancouver}
\bibliography{references}