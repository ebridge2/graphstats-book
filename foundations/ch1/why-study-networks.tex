\section{Why do we study networks?}
\label{sec:ch1:howstudy}

When you start thinking, you start seeing networks everywhere. The objects could be people, and the edges could be friendships. Or they could be computers, and the relationship could be the information they send to each other. You'd have airflight networks if you're working in air traffic control, where the edges are flights from city A to B; or if you're an epidemiologist studying disease, you could create an infection network. Neuroscientists can explore brain networks, which tell them about neurons and their relationships with each other, and computer scientists often use neural networks, which have become pillars in machine learning.

Networks can even be used to visualize ideas: in a Bayesian network, your nodes represent a set of variables and the relationship between them is their conditional dependencies. In a correlation network, your nodes represent variables, and the edges represent the correlation between those variables. You can have ecological networks, electrical networks, gene networks, and you could visualize your team's workflow with a network.

Even the way we think can be thought of as a network. Visualize a concept or object in your head. Maybe you could think about the food you had for breakfast this morning, or the city you live in. Now, think about the connections between those concepts and others. Maybe you had eggs and toast for breakfast. Eggs and toast are connected with a multitude of other concepts in your head: forks and silverware, kitchens, hunger, protein, carbohydrates, your morning routine, chickens, wheat, other breakfast foods, and an innumerable amount of other things. What you're doing right now is exploring a small part of the massive semantic network that lives in your head.

Network machine learning is a relatively new field. The vast majority of it has been invented (or discovered) after the year 2000, and many fundamental proofs have only been published recently. 

For the business-minded folks out there, this is an incredible time to learn about networks. Graph neural networks (GNNs) are becoming increasingly popular as deep learning and neural networks explode. This book provides the basic foundational concepts and intuition required to understand how, when, and why GNNs, or any other network machine learning tool, work.

Real-life applications also follow a general trend. You'll see academia spend a lot of time, usually 10-20 years, publishing proof-of-concept papers, discussing possible approaches to solve problems, and developing fundamental tools (usually informally, with somewhat messy code that exists in jupyter notebooks). Then, as the field of research starts maturing, companies and industry people start noticing these new academic tools. They find ways to apply them to make their product or service better, and easy-to-use packages like scikit-learn are developed to make these academic tools mainstream. Network machine learning is at a tipping point right now: its academic foundations have been built up over the past 10-20 years, and the tools for building and working with networks are now starting to move from academia to industry. Congratulations: you could get in early for a wave of application-focused network machine learning tools!
